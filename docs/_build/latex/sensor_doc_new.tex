%% Generated by Sphinx.
\def\sphinxdocclass{report}
\documentclass[letterpaper,10pt,english]{sphinxmanual}
\ifdefined\pdfpxdimen
   \let\sphinxpxdimen\pdfpxdimen\else\newdimen\sphinxpxdimen
\fi \sphinxpxdimen=.75bp\relax

\PassOptionsToPackage{warn}{textcomp}
\usepackage[utf8]{inputenc}
\ifdefined\DeclareUnicodeCharacter
% support both utf8 and utf8x syntaxes
  \ifdefined\DeclareUnicodeCharacterAsOptional
    \def\sphinxDUC#1{\DeclareUnicodeCharacter{"#1}}
  \else
    \let\sphinxDUC\DeclareUnicodeCharacter
  \fi
  \sphinxDUC{00A0}{\nobreakspace}
  \sphinxDUC{2500}{\sphinxunichar{2500}}
  \sphinxDUC{2502}{\sphinxunichar{2502}}
  \sphinxDUC{2514}{\sphinxunichar{2514}}
  \sphinxDUC{251C}{\sphinxunichar{251C}}
  \sphinxDUC{2572}{\textbackslash}
\fi
\usepackage{cmap}
\usepackage[T1]{fontenc}
\usepackage{amsmath,amssymb,amstext}
\usepackage{babel}



\usepackage{times}
\expandafter\ifx\csname T@LGR\endcsname\relax
\else
% LGR was declared as font encoding
  \substitutefont{LGR}{\rmdefault}{cmr}
  \substitutefont{LGR}{\sfdefault}{cmss}
  \substitutefont{LGR}{\ttdefault}{cmtt}
\fi
\expandafter\ifx\csname T@X2\endcsname\relax
  \expandafter\ifx\csname T@T2A\endcsname\relax
  \else
  % T2A was declared as font encoding
    \substitutefont{T2A}{\rmdefault}{cmr}
    \substitutefont{T2A}{\sfdefault}{cmss}
    \substitutefont{T2A}{\ttdefault}{cmtt}
  \fi
\else
% X2 was declared as font encoding
  \substitutefont{X2}{\rmdefault}{cmr}
  \substitutefont{X2}{\sfdefault}{cmss}
  \substitutefont{X2}{\ttdefault}{cmtt}
\fi


\usepackage[Bjarne]{fncychap}
\usepackage{sphinx}

\fvset{fontsize=\small}
\usepackage{geometry}

% Include hyperref last.
\usepackage{hyperref}
% Fix anchor placement for figures with captions.
\usepackage{hypcap}% it must be loaded after hyperref.
% Set up styles of URL: it should be placed after hyperref.
\urlstyle{same}

\usepackage{sphinxmessages}
\setcounter{tocdepth}{1}



\title{Sensor Documentation}
\date{Jun 23, 2019}
\release{0.0.6}
\author{Jason Gao}
\newcommand{\sphinxlogo}{\vbox{}}
\renewcommand{\releasename}{Release}
\makeindex
\begin{document}

\pagestyle{empty}
\sphinxmaketitle
\pagestyle{plain}
\tableofcontents
\pagestyle{normal}
\phantomsection\label{\detokenize{index::doc}}



\chapter{About project}
\label{\detokenize{index:about-project}}
This project is funded by the Defense Research and Development Canada (DRDC) and they are working alongside Carleton University to develop algorithms for radar and lidar sensors. The radars used in this project are Novelda X4,
Thor WFS30-K1, Ouster lidar and Advacam Minipix TPX3.

Contents:


\section{Radar Information}
\label{\detokenize{radar information:radar-information}}\label{\detokenize{radar information::doc}}

\subsection{About X4 radar}
\label{\detokenize{radar information:about-x4-radar}}
The X4 radars are IR-UWB and can work at frequencies ranging from 6 GHz to 10.2 GHz. The total number of bins that can be sampled is 1536.


\subsubsection{X4M300 Specs}
\label{\detokenize{radar information:x4m300-specs}}\begin{itemize}
\item {} 
Detection Time: 1.5 - 3.0 seconds

\item {} 
Range: 9.4 meters

\item {} 
Antenna: Tx for transmission and Rx for receiving

\item {} 
Baseband data output: 17 baseband/ssecond

\item {} 
System on chip: Novelda UWB X4

\end{itemize}


\subsubsection{X4M200 Specs}
\label{\detokenize{radar information:x4m200-specs}}\begin{itemize}
\item {} 
Detection Time: 3.0  - 5.0 seconds

\item {} 
Range: 5 meters

\item {} 
Antenna: Tx for transmission and Rx for receiving

\item {} 
Baseband data output: 17 baseband/ssecond

\item {} 
System on chip: Novelda UWB X4

\end{itemize}


\subsection{Configuring X4 radar}
\label{\detokenize{radar information:configuring-x4-radar}}\begin{enumerate}
\def\theenumi{\arabic{enumi}}
\def\labelenumi{\theenumi .}
\makeatletter\def\p@enumii{\p@enumi \theenumi .}\makeatother
\item {} 
Begin by initializing to default values using prebuilt function \sphinxstyleemphasis{x4driver\_init()}

\item {} 
Set PRF using function \sphinxstyleemphasis{x4driver\_set\_prf\_div(…)}

\end{enumerate}

\begin{sphinxadmonition}{note}{Note:}
The common PLL value of 243 MHz is divided by the arguemnent passed in to \sphinxstyleemphasis{x4driver\_set\_prf\_div(…)} to get a PRF value
\end{sphinxadmonition}

\begin{sphinxadmonition}{note}{Note:}
Make sure that when changing the PRF that frame length is shorter than 1/PRF and avoid sampling previous pulse when transmitting next pulse.
\end{sphinxadmonition}
\begin{enumerate}
\def\theenumi{\arabic{enumi}}
\def\labelenumi{\theenumi .}
\makeatletter\def\p@enumii{\p@enumi \theenumi .}\makeatother
\setcounter{enumi}{2}
\item {} 
Set DAC sweep range minimum and maximum using \sphinxstyleemphasis{x4driver\_set\_dac\_min()} and \sphinxstyleemphasis{x4driver\_set\_dac\_max()}

\item {} 
Set 0 reference using \sphinxstyleemphasis{x4driver\_set\_frame\_area\_offset()}

\item {} 
Set frame area using function \sphinxstyleemphasis{x4driver\_set\_frame\_area()} that takes two arguements, one for start of frame and one for end of frame.

\end{enumerate}


\subsection{Setting radar FPS}
\label{\detokenize{radar information:setting-radar-fps}}
To set the radar FPS the following parameters are required, PRF, iterations, pulse per step, dac max and dac min range as well as duty cycle.
\begin{equation*}
\begin{split}FPS = \frac{PRF}{iteration*pulse_per_step*(dac_max-dac_min+1)} * duty cycle\end{split}
\end{equation*}
Our Novelda radar is configured to a FPS of 17 pulse/second so if you wanted to change FPS then the above parameter would need to be changed.

\begin{sphinxadmonition}{note}{Note:}
The resulting FPS can be read using the built-in function \sphinxstyleemphasis{x4driver\_get\_fps()}.
\end{sphinxadmonition}


\subsubsection{Example pulse\_per\_step calculation}
\label{\detokenize{radar information:example-pulse-per-step-calculation}}\begin{itemize}
\item {} 
PRF: 16 MHz

\item {} 
X4\_duty\_cycle: 95\%

\item {} 
dac\_max: 1100

\item {} 
dac\_min: 949

\item {} 
iteration: 64

\item {} 
FPS: 17

\end{itemize}
\begin{equation*}
\begin{split}pulse\_per\_step &= \frac{PRF}{iteration*FPS*(dac_max-dac_min+1} * D \\
pulse\_per\_step  &= \frac{16 MHz}{64*17*150} * 0.95 \\
pulse\_per\_step  &= 87\end{split}
\end{equation*}

\section{X4 Radar}
\label{\detokenize{X4 radar:x4-radar}}\label{\detokenize{X4 radar::doc}}

\subsection{Parser for iq data}
\label{\detokenize{X4 radar:module-X4_parser}}\label{\detokenize{X4 radar:parser-for-iq-data}}\index{X4\_parser (module)@\spxentry{X4\_parser}\spxextra{module}}\index{iq\_data() (in module X4\_parser)@\spxentry{iq\_data()}\spxextra{in module X4\_parser}}

\begin{fulllineitems}
\phantomsection\label{\detokenize{X4 radar:X4_parser.iq_data}}\pysiglinewithargsret{\sphinxcode{\sphinxupquote{X4\_parser.}}\sphinxbfcode{\sphinxupquote{iq\_data}}}{\emph{filename}, \emph{csvname}}{}
Takes binary data file and iterates through in-phase (real) and quadrature (imaginary) values.
Data from range bins is taken and in-phase values are matched with quadrature values to be stored in a user defined .csv file.

Parameters:
\begin{quote}
\begin{description}
\item[{filename: str}] \leavevmode
The .dat binary file name.

\item[{csvname: str}] \leavevmode
User defined .csv file name

\end{description}
\end{quote}

Example:

\begin{sphinxVerbatim}[commandchars=\\\{\}]
\PYG{g+gp}{\PYGZgt{}\PYGZgt{}\PYGZgt{} }\PYG{n}{iq\PYGZus{}data}\PYG{p}{(}\PYG{l+s+s1}{\PYGZsq{}}\PYG{l+s+s1}{X4data.dat}\PYG{l+s+s1}{\PYGZsq{}}\PYG{p}{,}\PYG{l+s+s1}{\PYGZsq{}}\PYG{l+s+s1}{X4iq\PYGZus{}data}\PYG{l+s+s1}{\PYGZsq{}}\PYG{p}{)}
\PYG{g+gp}{\PYGZgt{}\PYGZgt{}\PYGZgt{} }\PYG{l+s+s1}{\PYGZsq{}}\PYG{l+s+s1}{converted}\PYG{l+s+s1}{\PYGZsq{}}
\end{sphinxVerbatim}

Return:
\begin{quote}

In-phase and quadrature pairs stored together in a .csv file.
\end{quote}

\end{fulllineitems}



\subsection{Parser for raw data}
\label{\detokenize{X4 radar:parser-for-raw-data}}

\begin{fulllineitems}
\pysiglinewithargsret{\sphinxcode{\sphinxupquote{X4\_parser.}}\sphinxbfcode{\sphinxupquote{raw\_data}}}{\emph{filename}, \emph{csvname}}{}
Takes raw data file and iterates through in-phase (real) and quadrature (imaginary) values.
Data from range bins is taken, and in-phase value are put apart from quadrature in a user defined .csv file.

Parameters:
\begin{quote}
\begin{description}
\item[{filename: str}] \leavevmode
The .dat binary file name.

\item[{csvname: str}] \leavevmode
User defined .csv file name

\end{description}
\end{quote}

Example:

\begin{sphinxVerbatim}[commandchars=\\\{\}]
\PYG{g+gp}{\PYGZgt{}\PYGZgt{}\PYGZgt{} }\PYG{n}{raw\PYGZus{}data}\PYG{p}{(}\PYG{l+s+s1}{\PYGZsq{}}\PYG{l+s+s1}{X4data.dat}\PYG{l+s+s1}{\PYGZsq{}}\PYG{p}{,}\PYG{l+s+s1}{\PYGZsq{}}\PYG{l+s+s1}{X4raw\PYGZus{}data}\PYG{l+s+s1}{\PYGZsq{}}\PYG{p}{)}
\PYG{g+gp}{\PYGZgt{}\PYGZgt{}\PYGZgt{} }\PYG{l+s+s1}{\PYGZsq{}}\PYG{l+s+s1}{converted}\PYG{l+s+s1}{\PYGZsq{}}
\end{sphinxVerbatim}

Return:
\begin{quote}

In-phase and quadrature stored separately in a .csv file.
\end{quote}

\end{fulllineitems}



\subsection{X4 Record and playback code}
\label{\detokenize{X4 radar:x4-record-and-playback-code}}
Target module: X4M200,X4M300,X4M03

Introduction:

XeThru modules support both RF and baseband data output. This is an example of radar raw data manipulation.
Developer can use Module Connecter API to read, record radar raw data, and also playback recorded data.

Command to run: \sphinxstyleemphasis{python X4\_record\_playback.py -d com4 -b -r}
\begin{itemize}
\item {} 
\sphinxstyleemphasis{-d com3} represents device name and can be found when starting Xethru Xplorer.

\item {} 
\sphinxstyleemphasis{-b} to use baseband to record.

\item {} 
\sphinxstyleemphasis{-r} to start recording.

\end{itemize}
\phantomsection\label{\detokenize{X4 radar:module-X4_record_playback}}\index{X4\_record\_playback (module)@\spxentry{X4\_record\_playback}\spxextra{module}}\index{clear\_buffer() (in module X4\_record\_playback)@\spxentry{clear\_buffer()}\spxextra{in module X4\_record\_playback}}

\begin{fulllineitems}
\phantomsection\label{\detokenize{X4 radar:X4_record_playback.clear_buffer}}\pysiglinewithargsret{\sphinxcode{\sphinxupquote{X4\_record\_playback.}}\sphinxbfcode{\sphinxupquote{clear\_buffer}}}{\emph{mc}}{}
Clears the frame buffer

Parameter:
\begin{quote}
\begin{description}
\item[{mc: object}] \leavevmode
module connector object

\end{description}
\end{quote}

\end{fulllineitems}

\index{main() (in module X4\_record\_playback)@\spxentry{main()}\spxextra{in module X4\_record\_playback}}

\begin{fulllineitems}
\phantomsection\label{\detokenize{X4 radar:X4_record_playback.main}}\pysiglinewithargsret{\sphinxcode{\sphinxupquote{X4\_record\_playback.}}\sphinxbfcode{\sphinxupquote{main}}}{}{}
Creates a parser with subcatergories.

Return:
\begin{quote}

A simple XEP plot of live feed from X4 radar.
\end{quote}

\end{fulllineitems}

\index{on\_file\_available() (in module X4\_record\_playback)@\spxentry{on\_file\_available()}\spxextra{in module X4\_record\_playback}}

\begin{fulllineitems}
\phantomsection\label{\detokenize{X4 radar:X4_record_playback.on_file_available}}\pysiglinewithargsret{\sphinxcode{\sphinxupquote{X4\_record\_playback.}}\sphinxbfcode{\sphinxupquote{on\_file\_available}}}{\emph{data\_type}, \emph{filename}}{}
Returns the file name that is available after recording.

Parameter:
\begin{quote}
\begin{description}
\item[{data\_type: str}] \leavevmode
data type of the recording file.

\item[{filename: str}] \leavevmode
file name of recording file.

\end{description}
\end{quote}

\end{fulllineitems}

\index{on\_meta\_file\_available() (in module X4\_record\_playback)@\spxentry{on\_meta\_file\_available()}\spxextra{in module X4\_record\_playback}}

\begin{fulllineitems}
\phantomsection\label{\detokenize{X4 radar:X4_record_playback.on_meta_file_available}}\pysiglinewithargsret{\sphinxcode{\sphinxupquote{X4\_record\_playback.}}\sphinxbfcode{\sphinxupquote{on\_meta\_file\_available}}}{\emph{session\_id}, \emph{meta\_filename}}{}
Returns the meta file name that is available after recording.

Parameters:
\begin{quote}
\begin{description}
\item[{session\_id: str}] \leavevmode
unique id to identify meta file

\item[{filename: str}] \leavevmode
file name of meta file.

\end{description}
\end{quote}

\end{fulllineitems}

\index{playback\_recording() (in module X4\_record\_playback)@\spxentry{playback\_recording()}\spxextra{in module X4\_record\_playback}}

\begin{fulllineitems}
\phantomsection\label{\detokenize{X4 radar:X4_record_playback.playback_recording}}\pysiglinewithargsret{\sphinxcode{\sphinxupquote{X4\_record\_playback.}}\sphinxbfcode{\sphinxupquote{playback\_recording}}}{\emph{meta\_filename}, \emph{baseband=False}}{}
Plays back the recording.

Parameters:
\begin{quote}
\begin{description}
\item[{meta\_filename: str}] \leavevmode
Name of meta file.

\item[{baseband: boolean}] \leavevmode
Check if recording with baseband iq data.

\end{description}
\end{quote}

\end{fulllineitems}

\index{reset() (in module X4\_record\_playback)@\spxentry{reset()}\spxextra{in module X4\_record\_playback}}

\begin{fulllineitems}
\phantomsection\label{\detokenize{X4 radar:X4_record_playback.reset}}\pysiglinewithargsret{\sphinxcode{\sphinxupquote{X4\_record\_playback.}}\sphinxbfcode{\sphinxupquote{reset}}}{\emph{device\_name}}{}
Resets the device profile and restarts the device

Parameter:
\begin{quote}
\begin{description}
\item[{device\_name: str}] \leavevmode
Identifies the device being used for recording with it’s port number.

\end{description}
\end{quote}

\end{fulllineitems}

\index{simple\_xep\_plot() (in module X4\_record\_playback)@\spxentry{simple\_xep\_plot()}\spxextra{in module X4\_record\_playback}}

\begin{fulllineitems}
\phantomsection\label{\detokenize{X4 radar:X4_record_playback.simple_xep_plot}}\pysiglinewithargsret{\sphinxcode{\sphinxupquote{X4\_record\_playback.}}\sphinxbfcode{\sphinxupquote{simple\_xep\_plot}}}{\emph{device\_name}, \emph{record=False}, \emph{baseband=False}}{}
Plots the recorded data.

Parameters:
\begin{quote}
\begin{description}
\item[{device\_name: str}] \leavevmode
port that device is connected to.

\item[{record: boolean}] \leavevmode
check if device is recording.

\item[{baseband: boolean}] \leavevmode
check if recording with baseband iq data.

\end{description}
\end{quote}

Return:
\begin{quote}

Simple plot of range over time.
\end{quote}

\end{fulllineitems}



\section{TI parser code}
\label{\detokenize{TI radar:module-TI_parser}}\label{\detokenize{TI radar:ti-parser-code}}\label{\detokenize{TI radar::doc}}\index{TI\_parser (module)@\spxentry{TI\_parser}\spxextra{module}}\index{readTIdata() (in module TI\_parser)@\spxentry{readTIdata()}\spxextra{in module TI\_parser}}

\begin{fulllineitems}
\phantomsection\label{\detokenize{TI radar:TI_parser.readTIdata}}\pysiglinewithargsret{\sphinxcode{\sphinxupquote{TI\_parser.}}\sphinxbfcode{\sphinxupquote{readTIdata}}}{\emph{filename}, \emph{csvname}}{}
Takes a .bin binary file and outputs the iq data to a csv file specified by csvname.
\begin{quote}\begin{description}
\item[{Parameter}] \leavevmode
\end{description}\end{quote}
\begin{description}
\item[{filename: str}] \leavevmode
file name of binary file.

\item[{csvname: str}] \leavevmode
csv file name that stores the iq data from binary file.

\end{description}
\begin{quote}\begin{description}
\item[{Example}] \leavevmode
\end{description}\end{quote}

\begin{sphinxVerbatim}[commandchars=\\\{\}]
\PYG{g+gp}{\PYGZgt{}\PYGZgt{}\PYGZgt{} }\PYG{n}{readTIdata}\PYG{p}{(}\PYG{l+s+s1}{\PYGZsq{}}\PYG{l+s+s1}{TIdata.bin}\PYG{l+s+s1}{\PYGZsq{}}\PYG{p}{,}\PYG{l+s+s1}{\PYGZsq{}}\PYG{l+s+s1}{TIdata}\PYG{l+s+s1}{\PYGZsq{}}\PYG{p}{)}
\PYG{g+gp}{\PYGZgt{}\PYGZgt{}\PYGZgt{} }\PYG{l+s+s1}{\PYGZsq{}}\PYG{l+s+s1}{converted}\PYG{l+s+s1}{\PYGZsq{}}
\end{sphinxVerbatim}
\begin{quote}\begin{description}
\item[{Returns}] \leavevmode


\end{description}\end{quote}

A csv file with the iq data taken from the binary file.

\end{fulllineitems}



\section{Test file}
\label{\detokenize{test:module-test}}\label{\detokenize{test:test-file}}\label{\detokenize{test::doc}}\index{test (module)@\spxentry{test}\spxextra{module}}\index{TestParser (class in test)@\spxentry{TestParser}\spxextra{class in test}}

\begin{fulllineitems}
\phantomsection\label{\detokenize{test:test.TestParser}}\pysiglinewithargsret{\sphinxbfcode{\sphinxupquote{class }}\sphinxcode{\sphinxupquote{test.}}\sphinxbfcode{\sphinxupquote{TestParser}}}{\emph{methodName='runTest'}}{}~\index{test\_TI() (test.TestParser method)@\spxentry{test\_TI()}\spxextra{test.TestParser method}}

\begin{fulllineitems}
\phantomsection\label{\detokenize{test:test.TestParser.test_TI}}\pysiglinewithargsret{\sphinxbfcode{\sphinxupquote{test\_TI}}}{}{}
Method to test if .bin binary file was converted successfully to .csv file with iq data put together.
\begin{quote}\begin{description}
\item[{Returns}] \leavevmode


\end{description}\end{quote}

converted

\end{fulllineitems}

\index{test\_iq() (test.TestParser method)@\spxentry{test\_iq()}\spxextra{test.TestParser method}}

\begin{fulllineitems}
\phantomsection\label{\detokenize{test:test.TestParser.test_iq}}\pysiglinewithargsret{\sphinxbfcode{\sphinxupquote{test\_iq}}}{}{}
Method to test if .dat binary file was converted successfully to .csv file with in-phase and quadrature
components together.
\begin{quote}\begin{description}
\item[{Returns}] \leavevmode


\end{description}\end{quote}

converted

\end{fulllineitems}

\index{test\_raw() (test.TestParser method)@\spxentry{test\_raw()}\spxextra{test.TestParser method}}

\begin{fulllineitems}
\phantomsection\label{\detokenize{test:test.TestParser.test_raw}}\pysiglinewithargsret{\sphinxbfcode{\sphinxupquote{test\_raw}}}{}{}
Method to test if .dat binary file was converted successfully to .csv file with in-phase and quadrature
component separated.
\begin{quote}\begin{description}
\item[{Returns}] \leavevmode


\end{description}\end{quote}

converted

\end{fulllineitems}


\end{fulllineitems}


\sphinxcode{\sphinxupquote{Convert X4 binary .dat file to csv}}
\sphinxcode{\sphinxupquote{Convert TI binary .bin file to csv}}


\renewcommand{\indexname}{Python Module Index}
\begin{sphinxtheindex}
\let\bigletter\sphinxstyleindexlettergroup
\bigletter{t}
\item\relax\sphinxstyleindexentry{test}\sphinxstyleindexpageref{test:\detokenize{module-test}}
\item\relax\sphinxstyleindexentry{TI\_parser}\sphinxstyleindexpageref{TI radar:\detokenize{module-TI_parser}}
\indexspace
\bigletter{x}
\item\relax\sphinxstyleindexentry{X4\_parser}\sphinxstyleindexpageref{X4 radar:\detokenize{module-X4_parser}}
\item\relax\sphinxstyleindexentry{X4\_record\_playback}\sphinxstyleindexpageref{X4 radar:\detokenize{module-X4_record_playback}}
\end{sphinxtheindex}

\renewcommand{\indexname}{Index}
\printindex
\end{document}