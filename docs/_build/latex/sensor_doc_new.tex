%% Generated by Sphinx.
\def\sphinxdocclass{report}
\documentclass[letterpaper,10pt,english]{sphinxmanual}
\ifdefined\pdfpxdimen
   \let\sphinxpxdimen\pdfpxdimen\else\newdimen\sphinxpxdimen
\fi \sphinxpxdimen=.75bp\relax

\PassOptionsToPackage{warn}{textcomp}
\usepackage[utf8]{inputenc}
\ifdefined\DeclareUnicodeCharacter
% support both utf8 and utf8x syntaxes
  \ifdefined\DeclareUnicodeCharacterAsOptional
    \def\sphinxDUC#1{\DeclareUnicodeCharacter{"#1}}
  \else
    \let\sphinxDUC\DeclareUnicodeCharacter
  \fi
  \sphinxDUC{00A0}{\nobreakspace}
  \sphinxDUC{2500}{\sphinxunichar{2500}}
  \sphinxDUC{2502}{\sphinxunichar{2502}}
  \sphinxDUC{2514}{\sphinxunichar{2514}}
  \sphinxDUC{251C}{\sphinxunichar{251C}}
  \sphinxDUC{2572}{\textbackslash}
\fi
\usepackage{cmap}
\usepackage[T1]{fontenc}
\usepackage{amsmath,amssymb,amstext}
\usepackage{babel}



\usepackage{times}
\expandafter\ifx\csname T@LGR\endcsname\relax
\else
% LGR was declared as font encoding
  \substitutefont{LGR}{\rmdefault}{cmr}
  \substitutefont{LGR}{\sfdefault}{cmss}
  \substitutefont{LGR}{\ttdefault}{cmtt}
\fi
\expandafter\ifx\csname T@X2\endcsname\relax
  \expandafter\ifx\csname T@T2A\endcsname\relax
  \else
  % T2A was declared as font encoding
    \substitutefont{T2A}{\rmdefault}{cmr}
    \substitutefont{T2A}{\sfdefault}{cmss}
    \substitutefont{T2A}{\ttdefault}{cmtt}
  \fi
\else
% X2 was declared as font encoding
  \substitutefont{X2}{\rmdefault}{cmr}
  \substitutefont{X2}{\sfdefault}{cmss}
  \substitutefont{X2}{\ttdefault}{cmtt}
\fi


\usepackage[Bjarne]{fncychap}
\usepackage{sphinx}

\fvset{fontsize=\small}
\usepackage{geometry}

% Include hyperref last.
\usepackage{hyperref}
% Fix anchor placement for figures with captions.
\usepackage{hypcap}% it must be loaded after hyperref.
% Set up styles of URL: it should be placed after hyperref.
\urlstyle{same}

\usepackage{sphinxmessages}
\setcounter{tocdepth}{1}



\title{Sensor Documentation}
\date{Jun 28, 2019}
\release{0.0.8}
\author{Jason Gao}
\newcommand{\sphinxlogo}{\vbox{}}
\renewcommand{\releasename}{Release}
\makeindex
\begin{document}

\pagestyle{empty}
\sphinxmaketitle
\pagestyle{plain}
\tableofcontents
\pagestyle{normal}
\phantomsection\label{\detokenize{index::doc}}



\chapter{About project}
\label{\detokenize{index:about-project}}
This project is funded by both the Defense Research and Development Canada (DRDC) and the National Research Council (NRC). These organizations are working alongside Carleton University students
to develop a pip installable library that contains all the parsers used to convert radar and lidar data to readable csv files.
The radars used in this project are the Novelda X4M03, X4M300, X4M200 and the TI-AWR1642. The lidar used is the Ouster OS1-16.

Contents:


\section{Radar Information}
\label{\detokenize{Radar information:radar-information}}\label{\detokenize{Radar information::doc}}

\subsection{About X4 radar}
\label{\detokenize{Radar information:about-x4-radar}}
The X4 radars are IR-UWB and can work at frequencies ranging from 6 GHz to 10.2 GHz. The total number of bins that can be sampled is 1536.


\subsubsection{X4M300 Specs}
\label{\detokenize{Radar information:x4m300-specs}}\begin{itemize}
\item {} 
Detection Time: 1.5 - 3.0 seconds

\item {} 
Range: 9.4 meters

\item {} 
Antenna: Tx for transmission and Rx for receiving

\item {} 
Baseband data output: 17 baseband/ssecond

\item {} 
System on chip: Novelda UWB X4

\end{itemize}


\subsubsection{X4M200 Specs}
\label{\detokenize{Radar information:x4m200-specs}}\begin{itemize}
\item {} 
Detection Time: 3.0  - 5.0 seconds

\item {} 
Range: 5 meters

\item {} 
Antenna: Tx for transmission and Rx for receiving

\item {} 
Baseband data output: 17 baseband/ssecond

\item {} 
System on chip: Novelda UWB X4

\end{itemize}


\subsection{Configuring X4 radar}
\label{\detokenize{Radar information:configuring-x4-radar}}\begin{enumerate}
\def\theenumi{\arabic{enumi}}
\def\labelenumi{\theenumi .}
\makeatletter\def\p@enumii{\p@enumi \theenumi .}\makeatother
\item {} 
Begin by initializing to default values using prebuilt function \sphinxstyleemphasis{x4driver\_init()}

\item {} 
Set PRF using function \sphinxstyleemphasis{x4driver\_set\_prf\_div(…)}

\end{enumerate}

\begin{sphinxadmonition}{note}{Note:}
The common PLL value of 243 MHz is divided by the arguemnent passed in to \sphinxstyleemphasis{x4driver\_set\_prf\_div(…)} to get a PRF value
\end{sphinxadmonition}

\begin{sphinxadmonition}{note}{Note:}
Make sure that when changing the PRF that frame length is shorter than 1/PRF and avoid sampling previous pulse when transmitting next pulse.
\end{sphinxadmonition}
\begin{enumerate}
\def\theenumi{\arabic{enumi}}
\def\labelenumi{\theenumi .}
\makeatletter\def\p@enumii{\p@enumi \theenumi .}\makeatother
\setcounter{enumi}{2}
\item {} 
Set DAC sweep range minimum and maximum using \sphinxstyleemphasis{x4driver\_set\_dac\_min()} and \sphinxstyleemphasis{x4driver\_set\_dac\_max()}

\item {} 
Set 0 reference using \sphinxstyleemphasis{x4driver\_set\_frame\_area\_offset()}

\item {} 
Set frame area using function \sphinxstyleemphasis{x4driver\_set\_frame\_area()} that takes two arguements, one for start of frame and one for end of frame.

\end{enumerate}


\subsection{Setting radar FPS}
\label{\detokenize{Radar information:setting-radar-fps}}
To set the radar FPS the following parameters are required, PRF, iterations, pulse per step, dac max and dac min range as well as duty cycle.
\begin{equation*}
\begin{split}FPS = \frac{PRF}{iteration*pulse_per_step*(dac_max-dac_min+1)} * duty cycle\end{split}
\end{equation*}
Our Novelda radar is configured to a FPS of 17 pulse/second so if you wanted to change FPS then the above parameter would need to be changed.

\begin{sphinxadmonition}{note}{Note:}
The resulting FPS can be read using the built-in function \sphinxstyleemphasis{x4driver\_get\_fps()}.
\end{sphinxadmonition}


\subsubsection{Example pulse\_per\_step calculation}
\label{\detokenize{Radar information:example-pulse-per-step-calculation}}\begin{itemize}
\item {} 
PRF: 16 MHz

\item {} 
X4\_duty\_cycle: 95\%

\item {} 
dac\_max: 1100

\item {} 
dac\_min: 949

\item {} 
iteration: 64

\item {} 
FPS: 17

\end{itemize}
\begin{equation*}
\begin{split}pulse\_per\_step &= \frac{PRF}{iteration*FPS*(dac_max-dac_min+1} * D \\
pulse\_per\_step  &= \frac{16 MHz}{64*17*150} * 0.95 \\
pulse\_per\_step  &= 87\end{split}
\end{equation*}

\section{Linux setup}
\label{\detokenize{Linux setup:linux-setup}}\label{\detokenize{Linux setup::doc}}
\sphinxstylestrong{To install Linux with Ubuntu v18.04 on a Windows PC, users must install the following files:}

\sphinxhref{https://www.virtualbox.org/wiki/Downloads}{VirutalBox Software}

\sphinxhref{https://ubuntu.com/download/desktop}{Ubuntu 18.04 download}


\subsection{Quick tips for linux beginners}
\label{\detokenize{Linux setup:quick-tips-for-linux-beginners}}\begin{itemize}
\item {} 
Set the network setting to use bridged adapter so Linux doesn’t share the same ip address as your Windows PC.

\item {} 
Insert guest addition CD image found in the \sphinxstyleemphasis{Devices} tab for auto-adjusted screen resolutions.

\item {} 
Use \sphinxstyleemphasis{\textasciitilde{}/ to cd to home directory}

\end{itemize}


\section{X4 Radar}
\label{\detokenize{X4 radar:x4-radar}}\label{\detokenize{X4 radar::doc}}

\subsection{Parser for iq data}
\label{\detokenize{X4 radar:parser-for-iq-data}}
Pass your .dat file from the recording into this function to generate a readable csv file with complex values as data.
.. automodule:: X4\_parser
\begin{quote}
\begin{quote}\begin{description}
\item[{members}] \leavevmode
iq\_data

\end{description}\end{quote}
\end{quote}


\subsection{Parser for raw data}
\label{\detokenize{X4 radar:parser-for-raw-data}}
Pass your .dat file from the recording into this function to generate a readable csv file with raw values as data.
.. automodule:: X4\_parser
\begin{quote}
\begin{quote}\begin{description}
\item[{noindex}] \leavevmode
\item[{members}] \leavevmode
raw\_data

\end{description}\end{quote}
\end{quote}


\subsection{X4 Record and playback code}
\label{\detokenize{X4 radar:x4-record-and-playback-code}}
Target module: X4M200,X4M300,X4M03

Introduction:

XeThru modules support both RF and baseband data output. This is an example of radar raw data manipulation.
Developer can use Module Connecter API to read, record radar raw data, and also playback recorded data.

Command to run: \sphinxstyleemphasis{python X4\_record\_playback.py -d com3-b -r}
\begin{itemize}
\item {} 
\sphinxstyleemphasis{-d com3} represents device name and can be found when starting Xethru Xplorer.

\item {} 
\sphinxstyleemphasis{-b} to use baseband to record, default is radio frequency.

\item {} 
\sphinxstyleemphasis{-r} to start recording.

\end{itemize}
\phantomsection\label{\detokenize{X4 radar:module-X4_record_playback}}\index{X4\_record\_playback (module)@\spxentry{X4\_record\_playback}\spxextra{module}}\index{clear\_buffer() (in module X4\_record\_playback)@\spxentry{clear\_buffer()}\spxextra{in module X4\_record\_playback}}

\begin{fulllineitems}
\phantomsection\label{\detokenize{X4 radar:X4_record_playback.clear_buffer}}\pysiglinewithargsret{\sphinxcode{\sphinxupquote{X4\_record\_playback.}}\sphinxbfcode{\sphinxupquote{clear\_buffer}}}{\emph{mc}}{}
Clears the frame buffer

Parameter:
\begin{description}
\item[{mc: object}] \leavevmode
module connector object

\end{description}

\end{fulllineitems}

\index{main() (in module X4\_record\_playback)@\spxentry{main()}\spxextra{in module X4\_record\_playback}}

\begin{fulllineitems}
\phantomsection\label{\detokenize{X4 radar:X4_record_playback.main}}\pysiglinewithargsret{\sphinxcode{\sphinxupquote{X4\_record\_playback.}}\sphinxbfcode{\sphinxupquote{main}}}{}{}
Creates a parser with subcatergories.

Return:

A simple XEP plot of live feed from X4 radar.

\end{fulllineitems}

\index{on\_file\_available() (in module X4\_record\_playback)@\spxentry{on\_file\_available()}\spxextra{in module X4\_record\_playback}}

\begin{fulllineitems}
\phantomsection\label{\detokenize{X4 radar:X4_record_playback.on_file_available}}\pysiglinewithargsret{\sphinxcode{\sphinxupquote{X4\_record\_playback.}}\sphinxbfcode{\sphinxupquote{on\_file\_available}}}{\emph{data\_type}, \emph{filename}}{}
Returns the file name that is available after recording.

Parameter:
\begin{description}
\item[{data\_type: str}] \leavevmode
data type of the recording file.

\item[{filename: str}] \leavevmode
file name of recording file.

\end{description}

\end{fulllineitems}

\index{on\_meta\_file\_available() (in module X4\_record\_playback)@\spxentry{on\_meta\_file\_available()}\spxextra{in module X4\_record\_playback}}

\begin{fulllineitems}
\phantomsection\label{\detokenize{X4 radar:X4_record_playback.on_meta_file_available}}\pysiglinewithargsret{\sphinxcode{\sphinxupquote{X4\_record\_playback.}}\sphinxbfcode{\sphinxupquote{on\_meta\_file\_available}}}{\emph{session\_id}, \emph{meta\_filename}}{}
Returns the meta file name that is available after recording.

Parameters:
\begin{description}
\item[{session\_id: str}] \leavevmode
unique id to identify meta file

\item[{filename: str}] \leavevmode
file name of meta file.

\end{description}

\end{fulllineitems}

\index{playback\_recording() (in module X4\_record\_playback)@\spxentry{playback\_recording()}\spxextra{in module X4\_record\_playback}}

\begin{fulllineitems}
\phantomsection\label{\detokenize{X4 radar:X4_record_playback.playback_recording}}\pysiglinewithargsret{\sphinxcode{\sphinxupquote{X4\_record\_playback.}}\sphinxbfcode{\sphinxupquote{playback\_recording}}}{\emph{meta\_filename}, \emph{baseband=False}}{}
Plays back the recording.

Parameters:
\begin{description}
\item[{meta\_filename: str}] \leavevmode
Name of meta file.

\item[{baseband: boolean}] \leavevmode
Check if recording with baseband iq data.

\end{description}

\end{fulllineitems}

\index{reset() (in module X4\_record\_playback)@\spxentry{reset()}\spxextra{in module X4\_record\_playback}}

\begin{fulllineitems}
\phantomsection\label{\detokenize{X4 radar:X4_record_playback.reset}}\pysiglinewithargsret{\sphinxcode{\sphinxupquote{X4\_record\_playback.}}\sphinxbfcode{\sphinxupquote{reset}}}{\emph{device\_name}}{}
Resets the device profile and restarts the device

Parameter:
\begin{description}
\item[{device\_name: str}] \leavevmode
Identifies the device being used for recording using it’s port number.

\end{description}

\end{fulllineitems}

\index{simple\_xep\_plot() (in module X4\_record\_playback)@\spxentry{simple\_xep\_plot()}\spxextra{in module X4\_record\_playback}}

\begin{fulllineitems}
\phantomsection\label{\detokenize{X4 radar:X4_record_playback.simple_xep_plot}}\pysiglinewithargsret{\sphinxcode{\sphinxupquote{X4\_record\_playback.}}\sphinxbfcode{\sphinxupquote{simple\_xep\_plot}}}{\emph{device\_name}, \emph{record=False}, \emph{baseband=False}}{}
Plots the recorded data.

Parameters:
\begin{description}
\item[{device\_name: str}] \leavevmode
port that device is connected to.

\item[{record: boolean}] \leavevmode
check if device is recording.

\item[{baseband: boolean}] \leavevmode
check if recording with baseband iq data.

\end{description}

Return:

Simple plot of range bin by amplitude.

\end{fulllineitems}



\section{TI parser code}
\label{\detokenize{TI radar:module-TI_parser}}\label{\detokenize{TI radar:ti-parser-code}}\label{\detokenize{TI radar::doc}}\index{TI\_parser (module)@\spxentry{TI\_parser}\spxextra{module}}\index{readTIdata() (in module TI\_parser)@\spxentry{readTIdata()}\spxextra{in module TI\_parser}}

\begin{fulllineitems}
\phantomsection\label{\detokenize{TI radar:TI_parser.readTIdata}}\pysiglinewithargsret{\sphinxcode{\sphinxupquote{TI\_parser.}}\sphinxbfcode{\sphinxupquote{readTIdata}}}{\emph{filename}, \emph{csvname}}{}
Takes a .bin binary file and outputs the iq data to a csv file specified by csvname.
\begin{quote}\begin{description}
\item[{Parameter}] \leavevmode
\end{description}\end{quote}
\begin{description}
\item[{filename: str}] \leavevmode
file name of binary file.

\item[{csvname: str}] \leavevmode
csv file name that stores the iq data from binary file.

\end{description}
\begin{quote}\begin{description}
\item[{Example}] \leavevmode
\end{description}\end{quote}

\begin{sphinxVerbatim}[commandchars=\\\{\}]
\PYG{g+gp}{\PYGZgt{}\PYGZgt{}\PYGZgt{} }\PYG{n}{readTIdata}\PYG{p}{(}\PYG{l+s+s1}{\PYGZsq{}}\PYG{l+s+s1}{TIdata.bin}\PYG{l+s+s1}{\PYGZsq{}}\PYG{p}{,}\PYG{l+s+s1}{\PYGZsq{}}\PYG{l+s+s1}{TIdata}\PYG{l+s+s1}{\PYGZsq{}}\PYG{p}{)}
\PYG{g+gp}{\PYGZgt{}\PYGZgt{}\PYGZgt{} }\PYG{l+s+s1}{\PYGZsq{}}\PYG{l+s+s1}{converted}\PYG{l+s+s1}{\PYGZsq{}}
\end{sphinxVerbatim}
\begin{quote}\begin{description}
\item[{Returns}] \leavevmode


\end{description}\end{quote}

Readable csv file containing complex data.

\end{fulllineitems}



\section{Ouster OS1 Lidar}
\label{\detokenize{Ouster lidar:ouster-os1-lidar}}\label{\detokenize{Ouster lidar::doc}}

\subsection{Specs}
\label{\detokenize{Ouster lidar:specs}}
\sphinxstyleemphasis{All below specs are for OS1-16 lidar that was used in this project}
- Works on channel 16, 64, 128.
- Maximum range of 120 meters.
- Field of view of 33.2 degree vertically and 360 degree horizontally.
- Sampling rate of 327,680 points/second.


\subsection{Setup lidar}
\label{\detokenize{Ouster lidar:setup-lidar}}\begin{enumerate}
\def\theenumi{\arabic{enumi}}
\def\labelenumi{\theenumi .}
\makeatletter\def\p@enumii{\p@enumi \theenumi .}\makeatother
\item {} 
Connect lidar interface box to router that supports Gigabit connection.

\item {} 
Connect lidar to lidar interface box via cable.

\item {} 
Determine the ip address your router gave the lidar when it connected to the network and jot it down.

\item {} 
Determine your linux ip adress by running \sphinxstyleemphasis{ifconfig} in terminal and jot it down.

\end{enumerate}


\subsection{Ouster Github}
\label{\detokenize{Ouster lidar:ouster-github}}
The following \sphinxhref{https://github.com/ouster-lidar/ouster\_example}{Github page} provides information on how to view raw data stream, visualize data and use a robot operating system (ROS) to save recorded data in a .bag file.
ROS commands can also replaying data in .bag files and convert .bag files to .csv files.

\begin{sphinxadmonition}{note}{Note:}
Some version of Linux running Ubuntu must be used. It is recommended to run Ubuntu 18.04 for best results. Follow instructions in \sphinxstyleemphasis{Ubuntu} page for more details on installing Linux with Ubuntu.
\end{sphinxadmonition}


\subsubsection{Ouster client}
\label{\detokenize{Ouster lidar:ouster-client}}
The Ouster client allows users to see the raw data stream that the lidar is collecting and sending to the specified ip address.
Instruction on building the client in Linux can be found here \sphinxhref{https://github.com/ouster-lidar/ouster\_example/tree/master/ouster\_client}{Building client}.


\paragraph{Running client}
\label{\detokenize{Ouster lidar:running-client}}\begin{enumerate}
\def\theenumi{\arabic{enumi}}
\def\labelenumi{\theenumi .}
\makeatletter\def\p@enumii{\p@enumi \theenumi .}\makeatother
\item {} 
cd /path/to/ouster\_client\_example

\item {} 
type \sphinxstyleemphasis{./ouster\_client\_example \textless{}os1\_hostname\textgreater{} \textless{}udp\_dest\_ip\textgreater{}}

\end{enumerate}

\sphinxstylestrong{\textless{}os1\_hostname\textgreater{}} is hostname/ip address of lidar.

\sphinxstylestrong{\textless{}udp\_data\_dest\_ip\textgreater{}} is the destination ip address the lidar sends data to. e.g. ip address from running \sphinxstyleemphasis{ifconfig}.


\subsubsection{Ouster visualization}
\label{\detokenize{Ouster lidar:ouster-visualization}}
The Ouster visualization is used for building a basic visualizer frame of collected lidar data. Instructions on building visualizer and it’s dependencies can be found here \sphinxhref{https://github.com/ouster-lidar/ouster\_example/tree/master/ouster\_viz}{Building visualizer}.
The visulaizer can be run in real time and with recorded data.


\paragraph{Running visualizer}
\label{\detokenize{Ouster lidar:running-visualizer}}\begin{enumerate}
\def\theenumi{\arabic{enumi}}
\def\labelenumi{\theenumi .}
\makeatletter\def\p@enumii{\p@enumi \theenumi .}\makeatother
\item {} 
cd /path/to/ouster\_viz/build

\item {} 
\sphinxstyleemphasis{./viz -m \textless{}frame\_size\textgreater{} \textless{}os1\_hostname\textgreater{} \textless{}udp\_data\_dest\_ip\textgreater{}}

\end{enumerate}

\sphinxstylestrong{\textless{}frame\_size\textgreater{}} is the size of the visualization frame and can ONLY be the following: 512x10, 512x20, 1024x10, 1024x20, 2048x10.

\sphinxstylestrong{\textless{}os1\_hostname\textgreater{}} is the ip address of lidar.

\sphinxstylestrong{\textless{}udp\_data\_ip\_dest\textgreater{}} is the ip address lidar sends data to.


\subsubsection{Ouster ROS}
\label{\detokenize{Ouster lidar:ouster-ros}}
\begin{sphinxadmonition}{note}{Note:}
For Ubuntu 18.04 users it is best to use \sphinxstylestrong{ROS Melodic} as \sphinxstylestrong{ROS Kinetic} (The ROS provided on the GitHub page) is only compatible with Ubuntu 16.04 and lower.
\end{sphinxadmonition}

Building the ROS Node can be found here \sphinxhref{https://github.com/ouster-lidar/ouster\_example/tree/master/ouster\_ros}{Buidling ROS Kinetic}.

For Ubuntu 16.04 users and lower: \sphinxhref{http://wiki.ros.org/kinetic/Installation/Ubuntu}{Installation of ROS Kinetic}

For Ubuntu 18.04 users: \sphinxhref{http://wiki.ros.org/melodic/Installation/Ubuntu}{Installation of ROS Melodic}

For new users to using ROS: \sphinxhref{http://wiki.ros.org/ROS/Tutorials}{ROS Tutorials}


\paragraph{Running ROS Node}
\label{\detokenize{Ouster lidar:running-ros-node}}
\begin{sphinxadmonition}{note}{Note:}
Before typing any commands make sure to always source the setup.bash file in your created ROS workspace otherwise it will return a error. The file can be sourced with the command \sphinxstyleemphasis{source /path/to/myworkspace/devel/setup.bash}.
\end{sphinxadmonition}

For recording lidar data:
\begin{enumerate}
\def\theenumi{\arabic{enumi}}
\def\labelenumi{\theenumi .}
\makeatletter\def\p@enumii{\p@enumi \theenumi .}\makeatother
\item {} 
\sphinxstyleemphasis{roslaunch ouster\_ros os1.launch os1\_hostname:=\textless{}os1\_hostname\textgreater{} os1\_udp\_dest:=\textless{}os1\_udp\_dest\textgreater{} lidar\_mode\textless{}:=\textless{}lidar\_mode\textgreater{}}. The option to visualize live data can be turned on by adding \sphinxstyleemphasis{viz:=true} to the roslaunch command.

\end{enumerate}

\sphinxstylestrong{\textless{}os1\_hostname\textgreater{} is the ip address of the lidar}

\sphinxstylestrong{\textless{}os1\_udp\_dest\textgreater{} is the ip address the lidar sends data to}

\sphinxstylestrong{\textless{}lidar\_mode\textgreater{} is the size of the lidar visualization frame}
\begin{enumerate}
\def\theenumi{\arabic{enumi}}
\def\labelenumi{\theenumi .}
\makeatletter\def\p@enumii{\p@enumi \theenumi .}\makeatother
\setcounter{enumi}{1}
\item {} 
\sphinxstyleemphasis{rosbag record -O \textless{}recorded\_\_bag\_filename\textgreater{} /os1\_node/imu\_packets /os1\_node/lidar\_packets} in a new terminal

\end{enumerate}

\sphinxstyleemphasis{/os1\_node/imu\_packets} and \sphinxstyleemphasis{/os1\_node/lidar\_packets} are your topic names that the lidar sends messages to via the node you built. These topic names can be changed to user preference.

\begin{sphinxadmonition}{note}{Note:}
DO NOT close the terminal with the roslaunch command open otherwise rosbag will crash.
\end{sphinxadmonition}

For replaying lidar data:
\begin{enumerate}
\def\theenumi{\arabic{enumi}}
\def\labelenumi{\theenumi .}
\makeatletter\def\p@enumii{\p@enumi \theenumi .}\makeatother
\item {} 
\sphinxstyleemphasis{roslaunch ouster\_ros os1.launch replay:=true os1\_hostname:=\textless{}os1\_hostname\textgreater{}}

\item {} 
In a \sphinxstylestrong{new} terminal run \sphinxstyleemphasis{rosbag play \textless{}bag\_filename\textgreater{}}

\end{enumerate}

\begin{sphinxadmonition}{note}{Note:}
DO NOT close the terminal with the roslaunch command open otherwise rosbag will crash.
\end{sphinxadmonition}

Converting data to csv file: Run \sphinxstyleemphasis{rostopic echo “topic name” -b “bag\_filename” -p \textgreater{} filename.csv}

\begin{sphinxadmonition}{note}{Note:}
To find topic names run the command \sphinxstyleemphasis{rosbag info \textless{}bag\_filename\textgreater{}}
\end{sphinxadmonition}


\section{Test file}
\label{\detokenize{Test:module-test}}\label{\detokenize{Test:test-file}}\label{\detokenize{Test::doc}}\index{test (module)@\spxentry{test}\spxextra{module}}\index{TestParser (class in test)@\spxentry{TestParser}\spxextra{class in test}}

\begin{fulllineitems}
\phantomsection\label{\detokenize{Test:test.TestParser}}\pysiglinewithargsret{\sphinxbfcode{\sphinxupquote{class }}\sphinxcode{\sphinxupquote{test.}}\sphinxbfcode{\sphinxupquote{TestParser}}}{\emph{methodName='runTest'}}{}~\index{test\_TI() (test.TestParser method)@\spxentry{test\_TI()}\spxextra{test.TestParser method}}

\begin{fulllineitems}
\phantomsection\label{\detokenize{Test:test.TestParser.test_TI}}\pysiglinewithargsret{\sphinxbfcode{\sphinxupquote{test\_TI}}}{}{}
Method to test if .bin binary file was converted successfully to .csv file with iq data put together.
\begin{quote}\begin{description}
\item[{Returns}] \leavevmode


\end{description}\end{quote}

Tells user if binary file was correctly converted to csv file.

\end{fulllineitems}

\index{test\_iq() (test.TestParser method)@\spxentry{test\_iq()}\spxextra{test.TestParser method}}

\begin{fulllineitems}
\phantomsection\label{\detokenize{Test:test.TestParser.test_iq}}\pysiglinewithargsret{\sphinxbfcode{\sphinxupquote{test\_iq}}}{}{}
Method to test if .dat binary file was converted successfully to .csv file with in-phase and quadrature
components together.
\begin{quote}\begin{description}
\item[{Returns}] \leavevmode


\end{description}\end{quote}

Tells user if binary file was correctly converted to csv file.

\end{fulllineitems}

\index{test\_raw() (test.TestParser method)@\spxentry{test\_raw()}\spxextra{test.TestParser method}}

\begin{fulllineitems}
\phantomsection\label{\detokenize{Test:test.TestParser.test_raw}}\pysiglinewithargsret{\sphinxbfcode{\sphinxupquote{test\_raw}}}{}{}
Method to test if .dat binary file was converted successfully to .csv file with in-phase and quadrature
component separated.
\begin{quote}\begin{description}
\item[{Returns}] \leavevmode


\end{description}\end{quote}

Tells user if binary file was correctly converted to csv file.

\end{fulllineitems}


\end{fulllineitems}


\sphinxcode{\sphinxupquote{Convert X4 binary .dat file to csv}}

\sphinxcode{\sphinxupquote{X4 data collection code}}

\sphinxcode{\sphinxupquote{Convert TI binary .bin file to csv}}

\sphinxcode{\sphinxupquote{Unit tests for radar parser}}


\renewcommand{\indexname}{Python Module Index}
\begin{sphinxtheindex}
\let\bigletter\sphinxstyleindexlettergroup
\bigletter{t}
\item\relax\sphinxstyleindexentry{test}\sphinxstyleindexpageref{Test:\detokenize{module-test}}
\item\relax\sphinxstyleindexentry{TI\_parser}\sphinxstyleindexpageref{TI radar:\detokenize{module-TI_parser}}
\indexspace
\bigletter{x}
\item\relax\sphinxstyleindexentry{X4\_record\_playback}\sphinxstyleindexpageref{X4 radar:\detokenize{module-X4_record_playback}}
\end{sphinxtheindex}

\renewcommand{\indexname}{Index}
\printindex
\end{document}