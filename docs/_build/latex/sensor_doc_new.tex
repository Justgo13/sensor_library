%% Generated by Sphinx.
\def\sphinxdocclass{report}
\documentclass[letterpaper,10pt,english]{sphinxmanual}
\ifdefined\pdfpxdimen
   \let\sphinxpxdimen\pdfpxdimen\else\newdimen\sphinxpxdimen
\fi \sphinxpxdimen=.75bp\relax

\PassOptionsToPackage{warn}{textcomp}
\usepackage[utf8]{inputenc}
\ifdefined\DeclareUnicodeCharacter
% support both utf8 and utf8x syntaxes
  \ifdefined\DeclareUnicodeCharacterAsOptional
    \def\sphinxDUC#1{\DeclareUnicodeCharacter{"#1}}
  \else
    \let\sphinxDUC\DeclareUnicodeCharacter
  \fi
  \sphinxDUC{00A0}{\nobreakspace}
  \sphinxDUC{2500}{\sphinxunichar{2500}}
  \sphinxDUC{2502}{\sphinxunichar{2502}}
  \sphinxDUC{2514}{\sphinxunichar{2514}}
  \sphinxDUC{251C}{\sphinxunichar{251C}}
  \sphinxDUC{2572}{\textbackslash}
\fi
\usepackage{cmap}
\usepackage[T1]{fontenc}
\usepackage{amsmath,amssymb,amstext}
\usepackage{babel}



\usepackage{times}
\expandafter\ifx\csname T@LGR\endcsname\relax
\else
% LGR was declared as font encoding
  \substitutefont{LGR}{\rmdefault}{cmr}
  \substitutefont{LGR}{\sfdefault}{cmss}
  \substitutefont{LGR}{\ttdefault}{cmtt}
\fi
\expandafter\ifx\csname T@X2\endcsname\relax
  \expandafter\ifx\csname T@T2A\endcsname\relax
  \else
  % T2A was declared as font encoding
    \substitutefont{T2A}{\rmdefault}{cmr}
    \substitutefont{T2A}{\sfdefault}{cmss}
    \substitutefont{T2A}{\ttdefault}{cmtt}
  \fi
\else
% X2 was declared as font encoding
  \substitutefont{X2}{\rmdefault}{cmr}
  \substitutefont{X2}{\sfdefault}{cmss}
  \substitutefont{X2}{\ttdefault}{cmtt}
\fi


\usepackage[Bjarne]{fncychap}
\usepackage{sphinx}

\fvset{fontsize=\small}
\usepackage{geometry}

% Include hyperref last.
\usepackage{hyperref}
% Fix anchor placement for figures with captions.
\usepackage{hypcap}% it must be loaded after hyperref.
% Set up styles of URL: it should be placed after hyperref.
\urlstyle{same}

\usepackage{sphinxmessages}
\setcounter{tocdepth}{1}



\title{CUDRDC Sensor Documentation}
\date{Jul 22, 2019}
\release{0.1.9}
\author{Jason Gao}
\newcommand{\sphinxlogo}{\vbox{}}
\renewcommand{\releasename}{Release}
\makeindex
\begin{document}

\pagestyle{empty}
\sphinxmaketitle
\pagestyle{plain}
\sphinxtableofcontents
\pagestyle{normal}
\phantomsection\label{\detokenize{index::doc}}



\chapter{About project}
\label{\detokenize{index:about-project}}
This project is funded by the Defense Research and Development Canada (DRDC) and the National Research Council (NRC). These organizations are working alongside Carleton University students
to develop a pip installable library that uses a GUI interface to run all the parsers used to convert radar and lidar data to readable csv files. The library GUI also contains methods for determining radar thresholds.
The radars used in this project are the Novelda X4M03, X4M300, X4M200 and the TSW1400. The lidar used is the Ouster OS1-16.

Contents:


\section{Radar Information}
\label{\detokenize{radar information:radar-information}}\label{\detokenize{radar information::doc}}

\subsection{Novelda X4}
\label{\detokenize{radar information:novelda-x4}}
The X4 radars are IR-UWB and can work at frequencies ranging from 6 GHz to 10.2 GHz. The total number of bins that can be sampled is 1536.


\subsubsection{Specifications}
\label{\detokenize{radar information:specifications}}

\paragraph{X4M300 Specs}
\label{\detokenize{radar information:x4m300-specs}}\begin{itemize}
\item {} 
Detection Time: 1.5 - 3.0 seconds

\item {} 
Range: 9.4 meters

\item {} 
Antenna: Tx for transmission and Rx for receiving

\item {} 
Baseband data output: 17 baseband/ssecond

\item {} 
System on chip: Novelda UWB X4

\end{itemize}


\paragraph{X4M200 Specs}
\label{\detokenize{radar information:x4m200-specs}}\begin{itemize}
\item {} 
Detection Time: 3.0  - 5.0 seconds

\item {} 
Range: 5 meters

\item {} 
Antenna: Tx for transmission and Rx for receiving

\item {} 
Baseband data output: 17 baseband/ssecond

\item {} 
System on chip: Novelda UWB X4

\end{itemize}


\subsubsection{Configuring X4 radar}
\label{\detokenize{radar information:configuring-x4-radar}}\begin{enumerate}
\def\theenumi{\arabic{enumi}}
\def\labelenumi{\theenumi .}
\makeatletter\def\p@enumii{\p@enumi \theenumi .}\makeatother
\item {} 
Begin by initializing to default values using prebuilt function \sphinxstyleemphasis{x4driver\_init()}

\item {} 
Set PRF using function \sphinxstyleemphasis{x4driver\_set\_prf\_div(…)}

\end{enumerate}

\begin{sphinxadmonition}{note}{Note:}
The common PLL value of 243 MHz is divided by the arguemnent passed in to \sphinxstyleemphasis{x4driver\_set\_prf\_div(…)} to get a PRF value
\end{sphinxadmonition}

\begin{sphinxadmonition}{note}{Note:}
Make sure that when changing the PRF that frame length is shorter than 1/PRF and avoid sampling previous pulse when transmitting next pulse.
\end{sphinxadmonition}
\begin{enumerate}
\def\theenumi{\arabic{enumi}}
\def\labelenumi{\theenumi .}
\makeatletter\def\p@enumii{\p@enumi \theenumi .}\makeatother
\setcounter{enumi}{2}
\item {} 
Set DAC sweep range minimum and maximum using \sphinxstyleemphasis{x4driver\_set\_dac\_min()} and \sphinxstyleemphasis{x4driver\_set\_dac\_max()}

\item {} 
Set 0 reference using \sphinxstyleemphasis{x4driver\_set\_frame\_area\_offset()}

\item {} 
Set frame area using function \sphinxstyleemphasis{x4driver\_set\_frame\_area()} that takes two arguements, one for start of frame and one for end of frame.

\end{enumerate}


\subsubsection{Setting radar FPS}
\label{\detokenize{radar information:setting-radar-fps}}
To set the radar FPS the following parameters are required, PRF, iterations, pulse per step, dac max and dac min range as well as duty cycle.
\begin{equation*}
\begin{split}FPS = \frac{PRF}{iteration*pulse_per_step*(dac_max-dac_min+1)} * duty cycle\end{split}
\end{equation*}
Our Novelda radar is configured to a FPS of 17 pulse/second so if you wanted to change FPS then the above parameter would need to be changed.

\begin{sphinxadmonition}{note}{Note:}
The resulting FPS can be read using the built-in function \sphinxstyleemphasis{x4driver\_get\_fps()}.
\end{sphinxadmonition}


\paragraph{Example pulse\_per\_step calculation}
\label{\detokenize{radar information:example-pulse-per-step-calculation}}\begin{itemize}
\item {} 
PRF: 16 MHz

\item {} 
X4\_duty\_cycle: 95\%

\item {} 
dac\_max: 1100

\item {} 
dac\_min: 949

\item {} 
iteration: 64

\item {} 
FPS: 17

\end{itemize}
\begin{equation*}
\begin{split}pulse\_per\_step &= \frac{PRF}{iteration*FPS*(dac_max-dac_min+1} * D \\
pulse\_per\_step  &= \frac{16 MHz}{64*17*150} * 0.95 \\
pulse\_per\_step  &= 87\end{split}
\end{equation*}

\subsection{TSW1400}
\label{\detokenize{radar information:tsw1400}}
The TSW1400 board is used to interface with TI radars.


\subsubsection{Specifications}
\label{\detokenize{radar information:id1}}\begin{itemize}
\item {} 
Operates using 5 V power source and is controlled by the SW7 switch.

\item {} 
11 LEDS used to indicate presence of power and state of FPGA.

\item {} 
Control of the TSW1400 is via USB cable to a Windows PC.

\end{itemize}


\subsubsection{Required softwares}
\label{\detokenize{radar information:required-softwares}}
\sphinxhref{https://drive.google.com/file/d/1yzbdIZaviq5P\_\_zpNsy\_TmBAXUXaU7Hg/view?usp=sharing}{Google Drive download}


\section{X4 Radar}
\label{\detokenize{X4 radar:x4-radar}}\label{\detokenize{X4 radar::doc}}

\subsection{Parser for iq data}
\label{\detokenize{X4 radar:module-X4_parser}}\label{\detokenize{X4 radar:parser-for-iq-data}}\index{X4\_parser (module)@\spxentry{X4\_parser}\spxextra{module}}\index{iq\_data() (in module X4\_parser)@\spxentry{iq\_data()}\spxextra{in module X4\_parser}}

\begin{fulllineitems}
\phantomsection\label{\detokenize{X4 radar:X4_parser.iq_data}}\pysiglinewithargsret{\sphinxcode{\sphinxupquote{X4\_parser.}}\sphinxbfcode{\sphinxupquote{iq\_data}}}{\emph{filename}, \emph{csvname}}{}
Reads in a binary file and data from range bins is taken and complex iq data is stored in a csv file specified by csvname.

Parameter:
\begin{quote}
\begin{description}
\item[{filename: str}] \leavevmode
The .dat binary file name.

\item[{csvname: str}] \leavevmode
User defined .csv file name

\end{description}
\end{quote}

Example:

\begin{sphinxVerbatim}[commandchars=\\\{\}]
\PYG{g+gp}{\PYGZgt{}\PYGZgt{}\PYGZgt{} }\PYG{n}{iq\PYGZus{}data}\PYG{p}{(}\PYG{l+s+s1}{\PYGZsq{}}\PYG{l+s+s1}{X4data.dat}\PYG{l+s+s1}{\PYGZsq{}}\PYG{p}{,}\PYG{l+s+s1}{\PYGZsq{}}\PYG{l+s+s1}{X4iq\PYGZus{}data}\PYG{l+s+s1}{\PYGZsq{}}\PYG{p}{)}
\PYG{g+gp}{\PYGZgt{}\PYGZgt{}\PYGZgt{} }\PYG{l+s+s1}{\PYGZsq{}}\PYG{l+s+s1}{converted}\PYG{l+s+s1}{\PYGZsq{}}
\end{sphinxVerbatim}

Returns:
\begin{quote}

Readable csv file containing complex values.
\end{quote}

\end{fulllineitems}



\subsection{Parser for raw data}
\label{\detokenize{X4 radar:parser-for-raw-data}}

\begin{fulllineitems}
\pysiglinewithargsret{\sphinxcode{\sphinxupquote{X4\_parser.}}\sphinxbfcode{\sphinxupquote{raw\_data}}}{\emph{filename}, \emph{csvname}}{}
Reads in a binary file and data from range bins is taken and raw data is stored in a csv file specified by csvname.

Parameters:
\begin{quote}
\begin{description}
\item[{filename: str}] \leavevmode
The .dat binary file name.

\item[{csvname: str}] \leavevmode
User defined .csv file name

\end{description}
\end{quote}

Example:

\begin{sphinxVerbatim}[commandchars=\\\{\}]
\PYG{g+gp}{\PYGZgt{}\PYGZgt{}\PYGZgt{} }\PYG{n}{raw\PYGZus{}data}\PYG{p}{(}\PYG{l+s+s1}{\PYGZsq{}}\PYG{l+s+s1}{X4data.dat}\PYG{l+s+s1}{\PYGZsq{}}\PYG{p}{,}\PYG{l+s+s1}{\PYGZsq{}}\PYG{l+s+s1}{X4raw\PYGZus{}data}\PYG{l+s+s1}{\PYGZsq{}}\PYG{p}{)}
\PYG{g+gp}{\PYGZgt{}\PYGZgt{}\PYGZgt{} }\PYG{l+s+s1}{\PYGZsq{}}\PYG{l+s+s1}{converted}\PYG{l+s+s1}{\PYGZsq{}}
\end{sphinxVerbatim}

Returns:
\begin{quote}

Readable csv files containing raw data.
\end{quote}

\end{fulllineitems}



\subsection{X4 Record and playback code}
\label{\detokenize{X4 radar:x4-record-and-playback-code}}
Target module: X4M200,X4M300,X4M03

Introduction:

XeThru modules support both RF and baseband data output. This is an example of radar raw data manipulation.
Developer can use Module Connecter API to read, record radar raw data, and also playback recorded data.

Command to run: \sphinxstyleemphasis{python X4\_record\_playback.py -d com3-b -r}
\begin{itemize}
\item {} 
\sphinxstyleemphasis{-d com3} represents device name and can be found when starting Xethru Xplorer.

\item {} 
\sphinxstyleemphasis{-b} to use baseband to record, default is radio frequency.

\item {} 
\sphinxstyleemphasis{-r} to start recording.

\end{itemize}

The pymoduleconnector library is used as an import in the X4 record and playback code and can be downloaded with the link below:

\sphinxhref{https://www.xethru.com/community/threads/module-connector-raspberry-pi.136/}{pymoduleconnector library download}

\phantomsection\label{\detokenize{X4 radar:module-X4_record_playback}}\index{X4\_record\_playback (module)@\spxentry{X4\_record\_playback}\spxextra{module}}\index{clear\_buffer() (in module X4\_record\_playback)@\spxentry{clear\_buffer()}\spxextra{in module X4\_record\_playback}}

\begin{fulllineitems}
\phantomsection\label{\detokenize{X4 radar:X4_record_playback.clear_buffer}}\pysiglinewithargsret{\sphinxcode{\sphinxupquote{X4\_record\_playback.}}\sphinxbfcode{\sphinxupquote{clear\_buffer}}}{\emph{mc}}{}
Clears the frame buffer

Parameter:
\begin{quote}
\begin{description}
\item[{mc: object}] \leavevmode
module connector object

\end{description}
\end{quote}

\end{fulllineitems}

\index{main() (in module X4\_record\_playback)@\spxentry{main()}\spxextra{in module X4\_record\_playback}}

\begin{fulllineitems}
\phantomsection\label{\detokenize{X4 radar:X4_record_playback.main}}\pysiglinewithargsret{\sphinxcode{\sphinxupquote{X4\_record\_playback.}}\sphinxbfcode{\sphinxupquote{main}}}{}{}
Creates a parser with subcatergories.

Return:
\begin{quote}

A simple XEP plot of live feed from X4 radar.
\end{quote}

\end{fulllineitems}

\index{on\_file\_available() (in module X4\_record\_playback)@\spxentry{on\_file\_available()}\spxextra{in module X4\_record\_playback}}

\begin{fulllineitems}
\phantomsection\label{\detokenize{X4 radar:X4_record_playback.on_file_available}}\pysiglinewithargsret{\sphinxcode{\sphinxupquote{X4\_record\_playback.}}\sphinxbfcode{\sphinxupquote{on\_file\_available}}}{\emph{data\_type}, \emph{filename}}{}
Returns the file name that is available after recording.

Parameters:
\begin{quote}
\begin{description}
\item[{data\_type: str}] \leavevmode
data type of the recording file.

\item[{filename: str}] \leavevmode
file name of recording file.

\end{description}
\end{quote}

\end{fulllineitems}

\index{on\_meta\_file\_available() (in module X4\_record\_playback)@\spxentry{on\_meta\_file\_available()}\spxextra{in module X4\_record\_playback}}

\begin{fulllineitems}
\phantomsection\label{\detokenize{X4 radar:X4_record_playback.on_meta_file_available}}\pysiglinewithargsret{\sphinxcode{\sphinxupquote{X4\_record\_playback.}}\sphinxbfcode{\sphinxupquote{on\_meta\_file\_available}}}{\emph{session\_id}, \emph{meta\_filename}}{}
Returns the meta file name that is available after recording.

Parameters:
\begin{quote}
\begin{description}
\item[{session\_id: str}] \leavevmode
unique id to identify meta file

\item[{filename: str}] \leavevmode
file name of meta file.

\end{description}
\end{quote}

\end{fulllineitems}

\index{playback\_recording() (in module X4\_record\_playback)@\spxentry{playback\_recording()}\spxextra{in module X4\_record\_playback}}

\begin{fulllineitems}
\phantomsection\label{\detokenize{X4 radar:X4_record_playback.playback_recording}}\pysiglinewithargsret{\sphinxcode{\sphinxupquote{X4\_record\_playback.}}\sphinxbfcode{\sphinxupquote{playback\_recording}}}{\emph{meta\_filename}, \emph{baseband=False}}{}
Plays back the recording.

Parameters:
\begin{quote}
\begin{description}
\item[{meta\_filename: str}] \leavevmode
Name of meta file.

\item[{baseband: boolean}] \leavevmode
Check if recording with baseband iq data.

\end{description}
\end{quote}

\end{fulllineitems}

\index{reset() (in module X4\_record\_playback)@\spxentry{reset()}\spxextra{in module X4\_record\_playback}}

\begin{fulllineitems}
\phantomsection\label{\detokenize{X4 radar:X4_record_playback.reset}}\pysiglinewithargsret{\sphinxcode{\sphinxupquote{X4\_record\_playback.}}\sphinxbfcode{\sphinxupquote{reset}}}{\emph{device\_name}}{}
Resets the device profile and restarts the device

Parameter:
\begin{quote}
\begin{description}
\item[{device\_name: str}] \leavevmode
Identifies the device being used for recording using it’s port number.

\end{description}
\end{quote}

\end{fulllineitems}

\index{simple\_xep\_plot() (in module X4\_record\_playback)@\spxentry{simple\_xep\_plot()}\spxextra{in module X4\_record\_playback}}

\begin{fulllineitems}
\phantomsection\label{\detokenize{X4 radar:X4_record_playback.simple_xep_plot}}\pysiglinewithargsret{\sphinxcode{\sphinxupquote{X4\_record\_playback.}}\sphinxbfcode{\sphinxupquote{simple\_xep\_plot}}}{\emph{device\_name}, \emph{record=False}, \emph{baseband=False}}{}
Plots the recorded data.

Parameters:
\begin{quote}
\begin{description}
\item[{device\_name: str}] \leavevmode
port that device is connected to.

\item[{record: boolean}] \leavevmode
check if device is recording.

\item[{baseband: boolean}] \leavevmode
check if recording with baseband iq data.

\end{description}
\end{quote}

Return:
\begin{quote}

Simple plot of range bin by amplitude.
\end{quote}

\end{fulllineitems}



\subsection{X4 Threshold detection}
\label{\detokenize{X4 radar:x4-threshold-detection}}
To use these functions first take the data recorded from the X4 radar and pass it into the iq\_data() function found in X4\_parser.py to get a comlex csv file. The file received will
be used wherever \sphinxstyleemphasis{filename} is an arguement.

\phantomsection\label{\detokenize{X4 radar:module-X4_threshold}}\index{X4\_threshold (module)@\spxentry{X4\_threshold}\spxextra{module}}\index{csv\_into\_list() (in module X4\_threshold)@\spxentry{csv\_into\_list()}\spxextra{in module X4\_threshold}}

\begin{fulllineitems}
\phantomsection\label{\detokenize{X4 radar:X4_threshold.csv_into_list}}\pysiglinewithargsret{\sphinxcode{\sphinxupquote{X4\_threshold.}}\sphinxbfcode{\sphinxupquote{csv\_into\_list}}}{\emph{filename}}{}
Converts data from a CSV file into a numpy array.

Data is collected from “XEP\_X4M200\_X4M300\_plot\_record\_playback.py”. Then 
.dat file is converted into a CSV file using the library.

Parameters:
\begin{quote}
\begin{description}
\item[{filename: str}] \leavevmode
Csv file name.

\end{description}
\end{quote}

Example:

\begin{sphinxVerbatim}[commandchars=\\\{\}]
\PYG{g+gp}{\PYGZgt{}\PYGZgt{}\PYGZgt{} }\PYG{n}{csv\PYGZus{}into\PYGZus{}list}\PYG{p}{(}\PYG{l+s+s2}{\PYGZdq{}}\PYG{l+s+s2}{Heli150040.csv}\PYG{l+s+s2}{\PYGZdq{}}\PYG{p}{)}
\PYG{g+gp}{\PYGZgt{}\PYGZgt{}\PYGZgt{} }\PYG{n}{array}\PYG{p}{(}\PYG{p}{[}\PYG{p}{[}\PYG{l+m+mf}{2.12019056e\PYGZhy{}04}\PYG{p}{,} \PYG{l+m+mf}{2.66481591e\PYGZhy{}04}\PYG{p}{,} \PYG{l+m+mf}{3.51781533e\PYGZhy{}04}\PYG{p}{,} \PYG{o}{.}\PYG{o}{.}\PYG{o}{.}\PYG{p}{,}
\PYG{g+go}{    2.51499279e\PYGZhy{}04, 2.63311858e\PYGZhy{}04, 2.49837321e\PYGZhy{}04],}
\PYG{g+go}{   [1.36839763e\PYGZhy{}03, 1.13654408e\PYGZhy{}03, 7.92290507e\PYGZhy{}04, ...,}
\PYG{g+go}{    3.49299137e\PYGZhy{}04, 1.30311682e\PYGZhy{}04, 1.87399805e\PYGZhy{}04],}
\PYG{g+go}{   [7.57068081e\PYGZhy{}04, 6.54611670e\PYGZhy{}04, 6.83445863e\PYGZhy{}04, ...,}
\PYG{g+go}{    6.95117789e\PYGZhy{}05, 2.17249015e\PYGZhy{}04, 3.85946482e\PYGZhy{}04],}
\PYG{g+go}{   ...,}
\PYG{g+go}{   [9.57032957e\PYGZhy{}04, 5.99047943e\PYGZhy{}04, 5.37280418e\PYGZhy{}04, ...,}
\PYG{g+go}{    2.95999810e\PYGZhy{}04, 2.42217122e\PYGZhy{}04, 3.25605109e\PYGZhy{}04],}
\PYG{g+go}{   [2.37110777e\PYGZhy{}03, 1.98986895e\PYGZhy{}03, 1.28061167e\PYGZhy{}03, ...,}
\PYG{g+go}{    2.95363991e\PYGZhy{}04, 1.50057104e\PYGZhy{}04, 2.33757752e\PYGZhy{}04],}
\PYG{g+go}{   [6.45687293e\PYGZhy{}04, 5.61334516e\PYGZhy{}04, 1.63285784e\PYGZhy{}04, ...,}
\PYG{g+go}{    6.64671904e\PYGZhy{}05, 2.17249015e\PYGZhy{}04, 1.93614790e\PYGZhy{}04]])}
\end{sphinxVerbatim}

Returns:
\begin{quote}

A numpy array of the data.
\end{quote}

\end{fulllineitems}

\index{distance\_finder() (in module X4\_threshold)@\spxentry{distance\_finder()}\spxextra{in module X4\_threshold}}

\begin{fulllineitems}
\phantomsection\label{\detokenize{X4 radar:X4_threshold.distance_finder}}\pysiglinewithargsret{\sphinxcode{\sphinxupquote{X4\_threshold.}}\sphinxbfcode{\sphinxupquote{distance\_finder}}}{\emph{filename}, \emph{estimate\_threshold}}{}
Converts the positive range bin/bins to a given distance in centimetres for a target.
Formula used for this is (bin)*5.25-18. Each range bin is 5.25 cm and starting offset is 18 cm.

Parameters:
\begin{quote}
\begin{description}
\item[{filename: str}] \leavevmode
Csv file name.

\item[{estimated\_threshold: float}] \leavevmode
Threshold to block out background noise.

\end{description}
\end{quote}

Example:

\begin{sphinxVerbatim}[commandchars=\\\{\}]
\PYG{g+gp}{\PYGZgt{}\PYGZgt{}\PYGZgt{} }\PYG{n}{distance\PYGZus{}finder}\PYG{p}{(}\PYG{l+s+s2}{\PYGZdq{}}\PYG{l+s+s2}{Heli150040.csv}\PYG{l+s+s2}{\PYGZdq{}}\PYG{p}{,}\PYG{l+m+mf}{0.02}\PYG{p}{)}
\PYG{g+gp}{\PYGZgt{}\PYGZgt{}\PYGZgt{} }\PYG{p}{[}\PYG{l+m+mf}{29.25}\PYG{p}{]}
\end{sphinxVerbatim}

Returns:
\begin{quote}

List of ranges in centimetres where all the possible targets above the threshold are located.
\end{quote}

\end{fulllineitems}

\index{noise\_power\_estimate() (in module X4\_threshold)@\spxentry{noise\_power\_estimate()}\spxextra{in module X4\_threshold}}

\begin{fulllineitems}
\phantomsection\label{\detokenize{X4 radar:X4_threshold.noise_power_estimate}}\pysiglinewithargsret{\sphinxcode{\sphinxupquote{X4\_threshold.}}\sphinxbfcode{\sphinxupquote{noise\_power\_estimate}}}{\emph{filename}, \emph{estimated\_threshold}}{}
Finds the noise power estimate.

Given the filename, the file is converted to an array. The average for the the array is taken excluding the points above the threshold and their respective guard cells.
The noise power estimate is then found by subtracting the positive and guard cells from the overall sum. When the noise power estimate is multiplied by the threshold factor, the threshold can be found.

Parameters:
\begin{quote}
\begin{description}
\item[{filename: str}] \leavevmode
Csv file name.

\item[{estimated\_threshold: float}] \leavevmode
Threshold to block out background noise.

\end{description}
\end{quote}

Example:

\begin{sphinxVerbatim}[commandchars=\\\{\}]
\PYG{g+gp}{\PYGZgt{}\PYGZgt{}\PYGZgt{} }\PYG{n}{noise\PYGZus{}power\PYGZus{}estimate}\PYG{p}{(}\PYG{l+s+s2}{\PYGZdq{}}\PYG{l+s+s2}{Heli150040.csv}\PYG{l+s+s2}{\PYGZdq{}}\PYG{p}{,}\PYG{l+m+mf}{0.02}\PYG{p}{)}
\PYG{g+gp}{\PYGZgt{}\PYGZgt{}\PYGZgt{} }\PYG{l+m+mf}{0.00045988029076158947}
\end{sphinxVerbatim}

Returns:
\begin{quote}

A number representing the noise power estimate.
\end{quote}

\end{fulllineitems}

\index{plot\_data() (in module X4\_threshold)@\spxentry{plot\_data()}\spxextra{in module X4\_threshold}}

\begin{fulllineitems}
\phantomsection\label{\detokenize{X4 radar:X4_threshold.plot_data}}\pysiglinewithargsret{\sphinxcode{\sphinxupquote{X4\_threshold.}}\sphinxbfcode{\sphinxupquote{plot\_data}}}{\emph{filename}}{}
Plots the data array for the 5th sample set.

Parameters:
\begin{quote}
\begin{description}
\item[{filename: str}] \leavevmode
Csv file name.

\end{description}
\end{quote}

Example:

\begin{sphinxVerbatim}[commandchars=\\\{\}]
\PYG{g+gp}{\PYGZgt{}\PYGZgt{}\PYGZgt{} }\PYG{n}{plot\PYGZus{}data}\PYG{p}{(}\PYG{l+s+s2}{\PYGZdq{}}\PYG{l+s+s2}{Heli150040.csv}\PYG{l+s+s2}{\PYGZdq{}}\PYG{p}{)}
\end{sphinxVerbatim}

Returns:
\begin{quote}

Graph showing the range bin with respect to their corresponding strength of signal.
\end{quote}

\end{fulllineitems}

\index{range\_finder() (in module X4\_threshold)@\spxentry{range\_finder()}\spxextra{in module X4\_threshold}}

\begin{fulllineitems}
\phantomsection\label{\detokenize{X4 radar:X4_threshold.range_finder}}\pysiglinewithargsret{\sphinxcode{\sphinxupquote{X4\_threshold.}}\sphinxbfcode{\sphinxupquote{range\_finder}}}{\emph{filename}, \emph{estimated\_threshold}}{}
Finds the range bin/bins from the radar data.

Given the filename, the file is converted to an array. The points above the given threshold are returned.

Parameters:
\begin{quote}
\begin{description}
\item[{filename: str}] \leavevmode
Csv file name

\item[{estimated\_threshold: float}] \leavevmode
Threshold to block out background noise.

\end{description}
\end{quote}

Example:

\begin{sphinxVerbatim}[commandchars=\\\{\}]
\PYG{g+gp}{\PYGZgt{}\PYGZgt{}\PYGZgt{} }\PYG{n}{range\PYGZus{}finder}\PYG{p}{(}\PYG{l+s+s2}{\PYGZdq{}}\PYG{l+s+s2}{Heli150040.csv}\PYG{l+s+s2}{\PYGZdq{}}\PYG{p}{,}\PYG{l+m+mf}{0.02}\PYG{p}{)}
\PYG{g+gp}{\PYGZgt{}\PYGZgt{}\PYGZgt{} }\PYG{p}{[}\PYG{l+m+mi}{9}\PYG{p}{]}
\end{sphinxVerbatim}

Returns:
\begin{quote}

A list of range bins that has signal strength values above the threshold.
\end{quote}

\end{fulllineitems}



\section{TSW parser code}
\label{\detokenize{TSW radar:module-TSW_IWR}}\label{\detokenize{TSW radar:tsw-parser-code}}\label{\detokenize{TSW radar::doc}}\index{TSW\_IWR (module)@\spxentry{TSW\_IWR}\spxextra{module}}\index{readTSWdata() (in module TSW\_IWR)@\spxentry{readTSWdata()}\spxextra{in module TSW\_IWR}}

\begin{fulllineitems}
\phantomsection\label{\detokenize{TSW radar:TSW_IWR.readTSWdata}}\pysiglinewithargsret{\sphinxcode{\sphinxupquote{TSW\_IWR.}}\sphinxbfcode{\sphinxupquote{readTSWdata}}}{\emph{filename}, \emph{csvname}}{}
Reads in a binary file and outputs the iq complex data to a csv file specified by csvname.

Parameter:
\begin{quote}
\begin{description}
\item[{filename: str}] \leavevmode
file name of binary file.

\item[{csvname: str}] \leavevmode
csv file name that stores the iq data from binary file.

\end{description}
\end{quote}

Example:

\begin{sphinxVerbatim}[commandchars=\\\{\}]
\PYG{g+gp}{\PYGZgt{}\PYGZgt{}\PYGZgt{} }\PYG{n}{readTSWdata}\PYG{p}{(}\PYG{l+s+s1}{\PYGZsq{}}\PYG{l+s+s1}{TIdata.bin}\PYG{l+s+s1}{\PYGZsq{}}\PYG{p}{,}\PYG{l+s+s1}{\PYGZsq{}}\PYG{l+s+s1}{TIdata}\PYG{l+s+s1}{\PYGZsq{}}\PYG{p}{)}
\PYG{g+gp}{\PYGZgt{}\PYGZgt{}\PYGZgt{} }\PYG{l+s+s1}{\PYGZsq{}}\PYG{l+s+s1}{converted}\PYG{l+s+s1}{\PYGZsq{}}
\end{sphinxVerbatim}

Return:
\begin{quote}

Readable csv file containing complex data.
\end{quote}

\end{fulllineitems}



\section{Library GUI}
\label{\detokenize{Library GUI:library-gui}}\label{\detokenize{Library GUI::doc}}
This GUI allows user to run parsers that are included in this library such as the ones for the Novelda X4 radars, TSW1642 radars and Ouster OS1-16 lidar. This GUI allows users to
open their desired file to read and with the use of buttons, read the desired data. The GUI also allows user to parse their binary files into a readable csv file.


\subsection{Functions}
\label{\detokenize{Library GUI:functions}}
Below, are the functions that are used and run in the GUI.

\phantomsection\label{\detokenize{Library GUI:module-library_gui}}\index{library\_gui (module)@\spxentry{library\_gui}\spxextra{module}}\index{command() (in module library\_gui)@\spxentry{command()}\spxextra{in module library\_gui}}

\begin{fulllineitems}
\phantomsection\label{\detokenize{Library GUI:library_gui.command}}\pysiglinewithargsret{\sphinxcode{\sphinxupquote{library\_gui.}}\sphinxbfcode{\sphinxupquote{command}}}{\emph{entry}}{}
Takes in a number of rows to be read from file and maps to list of ints.

Paramters:
\begin{quote}
\begin{description}
\item[{entry: int}] \leavevmode
The row numbers that will be output.

\end{description}
\end{quote}

Return:
\begin{quote}

mapped list of data
\end{quote}

\end{fulllineitems}

\index{convert\_TSW() (in module library\_gui)@\spxentry{convert\_TSW()}\spxextra{in module library\_gui}}

\begin{fulllineitems}
\phantomsection\label{\detokenize{Library GUI:library_gui.convert_TSW}}\pysiglinewithargsret{\sphinxcode{\sphinxupquote{library\_gui.}}\sphinxbfcode{\sphinxupquote{convert\_TSW}}}{}{}
Converts the TSW binary file to a readable csv file.

\end{fulllineitems}

\index{convert\_x4() (in module library\_gui)@\spxentry{convert\_x4()}\spxextra{in module library\_gui}}

\begin{fulllineitems}
\phantomsection\label{\detokenize{Library GUI:library_gui.convert_x4}}\pysiglinewithargsret{\sphinxcode{\sphinxupquote{library\_gui.}}\sphinxbfcode{\sphinxupquote{convert\_x4}}}{}{}
GUI window with button to allow conversion of X4 binary file to csv file.

Return:
\begin{quote}

Readable X4 csv file.
\end{quote}

\end{fulllineitems}

\index{imu\_multiple\_row() (in module library\_gui)@\spxentry{imu\_multiple\_row()}\spxextra{in module library\_gui}}

\begin{fulllineitems}
\phantomsection\label{\detokenize{Library GUI:library_gui.imu_multiple_row}}\pysiglinewithargsret{\sphinxcode{\sphinxupquote{library\_gui.}}\sphinxbfcode{\sphinxupquote{imu\_multiple\_row}}}{}{}
GUI window for reading IMU data of multiple rows

Return:
\begin{quote}

A textbox of the parameter data user wanted to read.
\end{quote}

\end{fulllineitems}

\index{imu\_row\_section() (in module library\_gui)@\spxentry{imu\_row\_section()}\spxextra{in module library\_gui}}

\begin{fulllineitems}
\phantomsection\label{\detokenize{Library GUI:library_gui.imu_row_section}}\pysiglinewithargsret{\sphinxcode{\sphinxupquote{library\_gui.}}\sphinxbfcode{\sphinxupquote{imu\_row\_section}}}{}{}
GUI window for reading IMU data of the row section

Return:
\begin{quote}

A textbox of the parameter data user wanted to read.
\end{quote}

\end{fulllineitems}

\index{imu\_single\_row() (in module library\_gui)@\spxentry{imu\_single\_row()}\spxextra{in module library\_gui}}

\begin{fulllineitems}
\phantomsection\label{\detokenize{Library GUI:library_gui.imu_single_row}}\pysiglinewithargsret{\sphinxcode{\sphinxupquote{library\_gui.}}\sphinxbfcode{\sphinxupquote{imu\_single\_row}}}{}{}
GUI window for reading IMU data of the single row

Return:
\begin{quote}

A textbox of the parameter data user wanted to read.
\end{quote}

\end{fulllineitems}

\index{instruction() (in module library\_gui)@\spxentry{instruction()}\spxextra{in module library\_gui}}

\begin{fulllineitems}
\phantomsection\label{\detokenize{Library GUI:library_gui.instruction}}\pysiglinewithargsret{\sphinxcode{\sphinxupquote{library\_gui.}}\sphinxbfcode{\sphinxupquote{instruction}}}{}{}
A set of instructions on how to use the program

\end{fulllineitems}

\index{lidar\_multiple\_row() (in module library\_gui)@\spxentry{lidar\_multiple\_row()}\spxextra{in module library\_gui}}

\begin{fulllineitems}
\phantomsection\label{\detokenize{Library GUI:library_gui.lidar_multiple_row}}\pysiglinewithargsret{\sphinxcode{\sphinxupquote{library\_gui.}}\sphinxbfcode{\sphinxupquote{lidar\_multiple\_row}}}{}{}
GUI window for reading lidar data of multiple rows

Return:
\begin{quote}

A textbox of the parameter data user wanted to read.
\end{quote}

\end{fulllineitems}

\index{lidar\_row\_section() (in module library\_gui)@\spxentry{lidar\_row\_section()}\spxextra{in module library\_gui}}

\begin{fulllineitems}
\phantomsection\label{\detokenize{Library GUI:library_gui.lidar_row_section}}\pysiglinewithargsret{\sphinxcode{\sphinxupquote{library\_gui.}}\sphinxbfcode{\sphinxupquote{lidar\_row\_section}}}{}{}
GUI window for reading lidar data of the row section

Return:
\begin{quote}

A textbox of the parameter data user wanted to read.
\end{quote}

\end{fulllineitems}

\index{lidar\_single\_row() (in module library\_gui)@\spxentry{lidar\_single\_row()}\spxextra{in module library\_gui}}

\begin{fulllineitems}
\phantomsection\label{\detokenize{Library GUI:library_gui.lidar_single_row}}\pysiglinewithargsret{\sphinxcode{\sphinxupquote{library\_gui.}}\sphinxbfcode{\sphinxupquote{lidar\_single\_row}}}{}{}
GUI window for reading lidar data in a single row

Return:
\begin{quote}

A textbox of the parameter data user wanted to read.
\end{quote}

\end{fulllineitems}

\index{open\_TSW\_bin() (in module library\_gui)@\spxentry{open\_TSW\_bin()}\spxextra{in module library\_gui}}

\begin{fulllineitems}
\phantomsection\label{\detokenize{Library GUI:library_gui.open_TSW_bin}}\pysiglinewithargsret{\sphinxcode{\sphinxupquote{library\_gui.}}\sphinxbfcode{\sphinxupquote{open\_TSW\_bin}}}{}{}
Opens the TSW binary file.

\end{fulllineitems}

\index{open\_x4\_bin() (in module library\_gui)@\spxentry{open\_x4\_bin()}\spxextra{in module library\_gui}}

\begin{fulllineitems}
\phantomsection\label{\detokenize{Library GUI:library_gui.open_x4_bin}}\pysiglinewithargsret{\sphinxcode{\sphinxupquote{library\_gui.}}\sphinxbfcode{\sphinxupquote{open\_x4\_bin}}}{}{}
Opens the Novelda X4 binary file for reading.

\end{fulllineitems}

\index{open\_x4\_csv() (in module library\_gui)@\spxentry{open\_x4\_csv()}\spxextra{in module library\_gui}}

\begin{fulllineitems}
\phantomsection\label{\detokenize{Library GUI:library_gui.open_x4_csv}}\pysiglinewithargsret{\sphinxcode{\sphinxupquote{library\_gui.}}\sphinxbfcode{\sphinxupquote{open\_x4\_csv}}}{}{}
Opens the Novelda X4 csv file for use in threshold functions.

\end{fulllineitems}

\index{openfile() (in module library\_gui)@\spxentry{openfile()}\spxextra{in module library\_gui}}

\begin{fulllineitems}
\phantomsection\label{\detokenize{Library GUI:library_gui.openfile}}\pysiglinewithargsret{\sphinxcode{\sphinxupquote{library\_gui.}}\sphinxbfcode{\sphinxupquote{openfile}}}{}{}
Opens the csv file for reading lidar and IMU packet parameters.

\end{fulllineitems}

\index{print\_list() (in module library\_gui)@\spxentry{print\_list()}\spxextra{in module library\_gui}}

\begin{fulllineitems}
\phantomsection\label{\detokenize{Library GUI:library_gui.print_list}}\pysiglinewithargsret{\sphinxcode{\sphinxupquote{library\_gui.}}\sphinxbfcode{\sphinxupquote{print\_list}}}{\emph{lst}}{}
Takes in an array and prints it to textbox.

Parameters:
\begin{quote}
\begin{description}
\item[{lst: list}] \leavevmode
A list object of the data that will be read to user.

\end{description}
\end{quote}

\end{fulllineitems}

\index{x4\_threshold() (in module library\_gui)@\spxentry{x4\_threshold()}\spxextra{in module library\_gui}}

\begin{fulllineitems}
\phantomsection\label{\detokenize{Library GUI:library_gui.x4_threshold}}\pysiglinewithargsret{\sphinxcode{\sphinxupquote{library\_gui.}}\sphinxbfcode{\sphinxupquote{x4\_threshold}}}{}{}
GUI window to run X4 threshold functions

Returns:
\begin{quote}

Radar threshold, noise and range estimates
\end{quote}

\end{fulllineitems}

\index{xyz\_calc() (in module library\_gui)@\spxentry{xyz\_calc()}\spxextra{in module library\_gui}}

\begin{fulllineitems}
\phantomsection\label{\detokenize{Library GUI:library_gui.xyz_calc}}\pysiglinewithargsret{\sphinxcode{\sphinxupquote{library\_gui.}}\sphinxbfcode{\sphinxupquote{xyz\_calc}}}{}{}
GUI window for reading the xyz coordinates gotten from lines of the lidar csv file.

Return:
\begin{quote}

A textbox of the xyz coordinates from rows of the lidar csv file.
\end{quote}

\end{fulllineitems}



\section{Linux setup}
\label{\detokenize{Linux setup:linux-setup}}\label{\detokenize{Linux setup::doc}}
\sphinxstylestrong{To install Linux with Ubuntu v18.04 on a Windows PC, users must install the following files:}

\sphinxhref{https://www.virtualbox.org/wiki/Downloads}{VirutalBox Software}

\sphinxhref{https://ubuntu.com/download/desktop}{Ubuntu 18.04 download}

\sphinxhref{https://itsfoss.com/install-ubuntu-dual-boot-mode-windows/}{Quick tutorial on installing Ubuntu}


\subsection{Quick tips for linux beginners}
\label{\detokenize{Linux setup:quick-tips-for-linux-beginners}}\begin{itemize}
\item {} 
Set the network setting to use bridged adapter so Linux doesn’t share the same ip address as your Windows PC.

\item {} 
Insert guest addition CD image found in the \sphinxstyleemphasis{Devices} tab for auto-adjusted screen resolutions.

\item {} 
Use \sphinxstyleemphasis{\textasciitilde{}/ to cd from home directory}

\end{itemize}


\section{Ouster OS1 Lidar}
\label{\detokenize{Ouster lidar:ouster-os1-lidar}}\label{\detokenize{Ouster lidar::doc}}

\subsection{Specs}
\label{\detokenize{Ouster lidar:specs}}
\sphinxstyleemphasis{All below specs are for the OS1-16 lidar that was used in this project}
- Works on channels 16, 64, 128.
- Maximum range of 120 meters.
- Field of view of 33.2 degree vertically and 360 degree horizontally.
- Sampling rate of 327,680 points/second.


\subsection{Setup lidar}
\label{\detokenize{Ouster lidar:setup-lidar}}\begin{enumerate}
\def\theenumi{\arabic{enumi}}
\def\labelenumi{\theenumi .}
\makeatletter\def\p@enumii{\p@enumi \theenumi .}\makeatother
\item {} 
Connect lidar interface box to router that supports Gigabit connection.

\item {} 
Connect lidar to lidar interface box via cable.

\item {} 
Determine the ip address your router gave the lidar when it connected to the network and jot it down.

\item {} 
Determine your linux ip adress by running \sphinxstyleemphasis{ifconfig} in terminal and jot it down.

\end{enumerate}


\subsection{Ouster Github}
\label{\detokenize{Ouster lidar:ouster-github}}
The following \sphinxhref{https://github.com/ouster-lidar/ouster\_example}{Github page} provides information on how to view raw data streams, visualize data and use a robot operating system (ROS) to save recorded data in a .bag file.
ROS commands can also replay data in .bag files and convert .bag files to .csv files.

\begin{sphinxadmonition}{note}{Note:}
Some version of Linux running Ubuntu must be used. It is recommended to run Ubuntu 18.04 for best results. Follow instructions in \sphinxstyleemphasis{Ubuntu} page for more details on installing Linux with Ubuntu.
\end{sphinxadmonition}


\subsubsection{Variable reference}
\label{\detokenize{Ouster lidar:variable-reference}}
\sphinxstylestrong{\textless{}os1\_hostname\textgreater{}} is the hostname/ip address of OS1-16 lidar.

\sphinxstylestrong{\textless{}udp\_data\_dest\_ip\textgreater{}} is the destination ip address the lidar sends data

\sphinxstylestrong{\textless{}frame\_size\textgreater{}} is the size of the visualization frame and can ONLY be the following: 512x10, 512x20, 1024x10, 1024x20, 2048x10.


\subsubsection{Ouster client}
\label{\detokenize{Ouster lidar:ouster-client}}
The Ouster client allows users to see the raw data stream that the lidar is collecting and sending to the specified ip address.
Instruction on building the client in Linux can be found here \sphinxhref{https://github.com/ouster-lidar/ouster\_example/tree/master/ouster\_client}{Building client}.


\paragraph{Running client}
\label{\detokenize{Ouster lidar:running-client}}\begin{enumerate}
\def\theenumi{\arabic{enumi}}
\def\labelenumi{\theenumi .}
\makeatletter\def\p@enumii{\p@enumi \theenumi .}\makeatother
\item {} 
cd /path/to/ouster\_client\_example

\item {} 
type \sphinxstyleemphasis{./ouster\_client\_example \textless{}os1\_hostname\textgreater{} \textless{}udp\_dest\_ip\textgreater{}}

\end{enumerate}


\subsubsection{Ouster visualization}
\label{\detokenize{Ouster lidar:ouster-visualization}}
The Ouster visualization is used for building a basic visualizer frame of collected lidar data. Instructions on building visualizer and it’s dependencies can be found here \sphinxhref{https://github.com/ouster-lidar/ouster\_example/tree/master/ouster\_viz}{Building visualizer}.
The visualizer can be run in real time or with recorded data.


\paragraph{Running visualizer}
\label{\detokenize{Ouster lidar:running-visualizer}}\begin{enumerate}
\def\theenumi{\arabic{enumi}}
\def\labelenumi{\theenumi .}
\makeatletter\def\p@enumii{\p@enumi \theenumi .}\makeatother
\item {} 
cd /path/to/ouster\_viz/build

\item {} 
\sphinxstyleemphasis{./viz -m \textless{}frame\_size\textgreater{} \textless{}os1\_hostname\textgreater{} \textless{}udp\_data\_dest\_ip\textgreater{}}

\end{enumerate}


\subsubsection{Ouster ROS}
\label{\detokenize{Ouster lidar:ouster-ros}}
\begin{sphinxadmonition}{note}{Note:}
For Ubuntu 18.04 users it is best to use \sphinxstylestrong{ROS Melodic} as \sphinxstylestrong{ROS Kinetic} (The ROS provided on the GitHub page) is only compatible with Ubuntu 16.04 and lower.
\end{sphinxadmonition}

Building the ROS Node can be found here \sphinxhref{https://github.com/ouster-lidar/ouster\_example/tree/master/ouster\_ros}{Building ROS Kinetic}.

For Ubuntu 16.04 users and lower: \sphinxhref{http://wiki.ros.org/kinetic/Installation/Ubuntu}{Installation of ROS Kinetic}

For Ubuntu 18.04 users: \sphinxhref{http://wiki.ros.org/melodic/Installation/Ubuntu}{Installation of ROS Melodic}

For new users to using ROS: \sphinxhref{http://wiki.ros.org/ROS/Tutorials}{ROS Tutorials}


\paragraph{Running ROS Node}
\label{\detokenize{Ouster lidar:running-ros-node}}
\begin{sphinxadmonition}{note}{Note:}
Before typing any commands make sure to always source the setup.bash file in your created ROS workspace otherwise it will return a error. The file can be sourced with the command \sphinxstyleemphasis{source /path/to/myworkspace/devel/setup.bash}.
\end{sphinxadmonition}

For recording lidar data:
\begin{enumerate}
\def\theenumi{\arabic{enumi}}
\def\labelenumi{\theenumi .}
\makeatletter\def\p@enumii{\p@enumi \theenumi .}\makeatother
\item {} 
\sphinxstyleemphasis{roslaunch ouster\_ros os1.launch os1\_hostname:=\textless{}os1\_hostname\textgreater{} os1\_udp\_dest:=\textless{}os1\_udp\_dest\textgreater{} lidar\_mode\textless{}:=\textless{}lidar\_mode\textgreater{}}. The option to visualize live data can be turned on by adding \sphinxstyleemphasis{viz:=true} to the roslaunch command.

\item {} 
\sphinxstyleemphasis{rosbag record -O \textless{}recorded\_\_bag\_filename\textgreater{} /os1\_node/imu\_packets /os1\_node/lidar\_packets} in a new terminal

\end{enumerate}

\sphinxstyleemphasis{/os1\_node/imu\_packets} and \sphinxstyleemphasis{/os1\_node/lidar\_packets} are your topic names that the lidar sends messages to via the node you built. These topic names can be changed to user preference.

\begin{sphinxadmonition}{note}{Note:}
DO NOT close the terminal with the roslaunch command open otherwise rosbag will crash.
\end{sphinxadmonition}

For replaying lidar data:
\begin{enumerate}
\def\theenumi{\arabic{enumi}}
\def\labelenumi{\theenumi .}
\makeatletter\def\p@enumii{\p@enumi \theenumi .}\makeatother
\item {} 
\sphinxstyleemphasis{roslaunch ouster\_ros os1.launch replay:=true os1\_hostname:=\textless{}os1\_hostname\textgreater{}}

\item {} 
In a \sphinxstylestrong{new} terminal run \sphinxstyleemphasis{rosbag play \textless{}bag\_filename\textgreater{}}

\end{enumerate}

\begin{sphinxadmonition}{note}{Note:}
DO NOT close the terminal with the roslaunch command open otherwise rosbag will crash.
\end{sphinxadmonition}

Converting data to csv file: Run \sphinxstyleemphasis{rostopic echo “topic name” -b “bag\_filename” -p \textgreater{} filename.csv}

\begin{sphinxadmonition}{note}{Note:}
To find topic names run the command \sphinxstyleemphasis{rosbag info \textless{}bag\_filename\textgreater{}}
\end{sphinxadmonition}


\section{Test file}
\label{\detokenize{test:module-test}}\label{\detokenize{test:test-file}}\label{\detokenize{test::doc}}\index{test (module)@\spxentry{test}\spxextra{module}}\index{TestParser (class in test)@\spxentry{TestParser}\spxextra{class in test}}

\begin{fulllineitems}
\phantomsection\label{\detokenize{test:test.TestParser}}\pysiglinewithargsret{\sphinxbfcode{\sphinxupquote{class }}\sphinxcode{\sphinxupquote{test.}}\sphinxbfcode{\sphinxupquote{TestParser}}}{\emph{methodName='runTest'}}{}~\index{X4\_Threshold\_bin\_to\_distance() (test.TestParser method)@\spxentry{X4\_Threshold\_bin\_to\_distance()}\spxextra{test.TestParser method}}

\begin{fulllineitems}
\phantomsection\label{\detokenize{test:test.TestParser.X4_Threshold_bin_to_distance}}\pysiglinewithargsret{\sphinxbfcode{\sphinxupquote{X4\_Threshold\_bin\_to\_distance}}}{}{}
Method to test if range bin to distance converted correctly

\end{fulllineitems}

\index{X4\_Threshold\_noise\_estimate() (test.TestParser method)@\spxentry{X4\_Threshold\_noise\_estimate()}\spxextra{test.TestParser method}}

\begin{fulllineitems}
\phantomsection\label{\detokenize{test:test.TestParser.X4_Threshold_noise_estimate}}\pysiglinewithargsret{\sphinxbfcode{\sphinxupquote{X4\_Threshold\_noise\_estimate}}}{}{}
Method to test if noise estimate was calculated properly

\end{fulllineitems}

\index{X4\_Threshold\_range\_finder() (test.TestParser method)@\spxentry{X4\_Threshold\_range\_finder()}\spxextra{test.TestParser method}}

\begin{fulllineitems}
\phantomsection\label{\detokenize{test:test.TestParser.X4_Threshold_range_finder}}\pysiglinewithargsret{\sphinxbfcode{\sphinxupquote{X4\_Threshold\_range\_finder}}}{}{}
Method to test if correct range bin was gotten from running function on csv file.

\end{fulllineitems}

\index{test\_TI() (test.TestParser method)@\spxentry{test\_TI()}\spxextra{test.TestParser method}}

\begin{fulllineitems}
\phantomsection\label{\detokenize{test:test.TestParser.test_TI}}\pysiglinewithargsret{\sphinxbfcode{\sphinxupquote{test\_TI}}}{}{}
Method to test if .bin binary file was converted successfully to .csv file with iq data put together.

\end{fulllineitems}

\index{test\_iq() (test.TestParser method)@\spxentry{test\_iq()}\spxextra{test.TestParser method}}

\begin{fulllineitems}
\phantomsection\label{\detokenize{test:test.TestParser.test_iq}}\pysiglinewithargsret{\sphinxbfcode{\sphinxupquote{test\_iq}}}{}{}
Method to test if .dat binary file was converted successfully to .csv file with in-phase and quadrature
components together.

\end{fulllineitems}

\index{test\_raw() (test.TestParser method)@\spxentry{test\_raw()}\spxextra{test.TestParser method}}

\begin{fulllineitems}
\phantomsection\label{\detokenize{test:test.TestParser.test_raw}}\pysiglinewithargsret{\sphinxbfcode{\sphinxupquote{test\_raw}}}{}{}
Method to test if .dat binary file was converted successfully to .csv file with in-phase and quadrature
component separated.

\end{fulllineitems}


\end{fulllineitems}



\renewcommand{\indexname}{Python Module Index}
\begin{sphinxtheindex}
\let\bigletter\sphinxstyleindexlettergroup
\bigletter{l}
\item\relax\sphinxstyleindexentry{library\_gui}\sphinxstyleindexpageref{Library GUI:\detokenize{module-library_gui}}
\indexspace
\bigletter{t}
\item\relax\sphinxstyleindexentry{test}\sphinxstyleindexpageref{test:\detokenize{module-test}}
\item\relax\sphinxstyleindexentry{TSW\_IWR}\sphinxstyleindexpageref{TSW radar:\detokenize{module-TSW_IWR}}
\indexspace
\bigletter{x}
\item\relax\sphinxstyleindexentry{X4\_parser}\sphinxstyleindexpageref{X4 radar:\detokenize{module-X4_parser}}
\item\relax\sphinxstyleindexentry{X4\_record\_playback}\sphinxstyleindexpageref{X4 radar:\detokenize{module-X4_record_playback}}
\item\relax\sphinxstyleindexentry{X4\_threshold}\sphinxstyleindexpageref{X4 radar:\detokenize{module-X4_threshold}}
\end{sphinxtheindex}

\renewcommand{\indexname}{Index}
\printindex
\end{document}